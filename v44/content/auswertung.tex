\section{Auswertung}
\label{sec:Auswertung}

\subsection{Bestimmung der Halbwertsbreite und der maximalen Intensität}
Die Halbwertsbreite und die maximale Intensität des Strahls kann mithilfe der Messdaten des Detektorscans bestimmt werden.
Dazu wird an die in \autoref{fig:DScan} abgebildeten Messdaten eine Gaußfunktion der Form
\begin{equation*}
  I(\alpha) = \frac{I_0}{\sqrt{2\symup{\pi}\sigma^2}}\cdot \symup{exp}\left(-\frac{(\alpha - \alpha_0)^2}{2\sigma^2}\right) + B
\end{equation*} 
angepasst. 
\begin{figure}
  \centering
  \includegraphics[width = .7\textwidth]{DScan.pdf}
  \caption{Messdaten des Detektorscans und mittels \texttt{scipy} \cite{scipy} angepasste Gaußfunktion. Es sind Markierungen zur Bestimmung der Halbwertsbreite eingezeichnet.}
  \label{fig:DScan}
\end{figure}
Der Fit wird mittels der \texttt{python} Bibliothek \texttt{scipy} \cite{scipy} durchgeführt.
Die freien Parameter ergeben sich zu
\begin{align*}
  \alpha_0 &= \qty{-2.83 +-0.47e-3}{\degree} \\
  \sigma &= \qty{3.78 +- 0.05e-2}{\degree} \\
  I_0 &= \num{1.61 +- 0.02e4} \\
  B &= \num{1.20 +- 0.43e3}.
\end{align*}
Die Halbwertsbreite (FWHM) lässt sich an der Höhe des Graphen ablesen, an welcher genau der halb Wert des Maximums erreicht wird. 
Es ergibt sich der Wert 
\begin{equation*}
  \text{FWHM} = \qty{0.091}{\degree}.
\end{equation*}


\subsection{Bestimmung der Strahlbreite}
Die Strahlbreite lässt sich aus den Messdaten des ersten durchgeführten Z-Scans ermitteln. Dazu werden die in \autoref{fig:ZScan} Messdaten betrachtet.
Die Strahlbreite entspricht der Breite der steil abfallenden Flanke des Graphen zwischen den beiden Plateubereichen.
\begin{figure}
  \centering
  \includegraphics[width = .7\textwidth]{ZScan.pdf}
  \caption{Messdaten des ersten Z-Scans. Die Breite des Bereiches zwischen den Plateubereichen entspricht der Strahlbreite.}
  \label{fig:ZScan}
\end{figure}
Es ergibt sich eine Strahlbreite von $d_0 \approx \qty{0.12}{\milli\metre}$. 

\subsection{Ermittlung des Geometriewinkels}
Trifft der Röntgenstrahl in einem kleinen Winkel auf die Probe, überschreitet die effektive Breites des Strahs die Probenlänge und ein 
gewisser Anteil der Intensität kann nicht reflektiert werden. Um diesen systematischen Effekt zu korrigieren, wird der Geometriewinkel bestimmt, bis zu welchen dieser 
Effekt auftritt. Er lässt sich anhand der Breite der in einem Rocking-Curve-Scan aufgenommenen Messkurve ermitteln.
Die entsprechenden Messdaten sind in \autoref{fig:RockingScan} abgebildet.
\begin{figure}
  \centering
  \includegraphics[width = .7\textwidth]{RockingScan.pdf}
  \caption{Rocking-Curve-Scan zur Bestimmung des Geometriewinkels.}
  \label{fig:RockingScan}
\end{figure}
Der Geometriewinkel ergibt sich zu $\alpha_\text{G} = \qty{0.44}{\degree}$. Mit der zuvor bestimmten Strahlbreite $d_0$ und der Probenlänge $D = \qty{20}{\milli\metre}$ bestimmt 
sich der Theoriewert des Geometriewinkels nach \autoref{eqn:a_g} zu $\alpha_\text{G, Theorie} = \qty{0.34}{\degree}$.

\subsection{Bestimmung der Dispersion und Rauigkeit des Siliziumwafers}
Um die Oberflächen- und Schichtstruktur des mit Polysterol beschichteten Siliziumwafers bestimmen zu können werden ein Reflektivitäts-Scan und ein Diffuser-Scan durchgeführt.
Zweiterer wird benötigt, um die zuvor gemessenen Reflektivität um den Anteil der gestreuten Intensität korrigieren zu können. Die Messdaten beider Scans und die um die gestreute 
Intensität korrigierten Werte sind in \autoref{fig:Reflek1} abgebildet.
\begin{figure}
  \centering
  \includegraphics[width = .7\textwidth]{Reflek1.pdf}
  \caption{Es sind die Messdaten des Reflektivitäts-Scans und des diffusen Scans zu sehen, sowie die Differenz der beiden Messreihen.}
  \label{fig:Reflek1}
\end{figure}
Im Folgenden wird anstatt der gemessenen Intensität, die Reflektivität $R$ betrachtet. Diese ergibt sich auf Grund der Messzeit von $\qty{5}{\second}$ zu 
\begin{equation*}
  R = \frac{I}{5I_0}.
\end{equation*}
Des Weiteren wird die gemessene Reflektivität nach \autoref{eq:Geometriefaktor} um den Geometriefaktor korrigiert.
Die Fresnelreflektivität einer ideal glatten Polysteroloberfläche kann durch \autoref{eq:Reflektivitaet} genähert werden. Der kritische Winkel $\alpha_\text{c}$ errechnet sich mit
\autoref{eq:alpha_tot} und $r_e\rho = \qty{20e14}{\metre^{-2}}$ \cite{V44} zu $\alpha_\text{c} = \qty{0.223}{\degree}$. Die Wellenlänge der Röntgenstrahlung ist die 
Wellenlänge der $K_\alpha$-Linie von Kupfer: $\lambda = \qty{1.54e-10}{\metre}$.
Die korrigierten Messwerte der Reflektivität sind in \autoref{fig:Reflek2} zusammen mit der Fresnelreflektivität aufgetragen.
\begin{figure}
  \centering
  \includegraphics[width = .7\textwidth]{Reflek2.pdf}
  \caption{Plot.}
  \label{fig:Reflek2}
\end{figure}
Darüber hinaus sind die periodischen Minima der Kiessigoszillation eingezeichnet, anhand welcher sich die Schichtdicke des Polysterolfilms ermitteln lässt.
Dazu wird der Abstand zwischen den Minima benötigt. Hierzu wird der Mittelwert und die Standardabweichung der Abstände der markierten Messwerte gebildet. Es ergibt sich 
$\symup{\Delta}\alpha = \qty{4.96 +- 0.46e-2}{\degree}$. Mit \autoref{eq:Schichtdicke} folgt die Schichtdicke $d = \qty{8.89 +- 0.82e-8}{\metre}$.

Da es sich bei der Probe um ein Mehrschichsystem aus Polysterolfilm und Siliziumwafer handelt, können über den Parrattalgorithmus (\eqref{eq:Parrat})
die Dispersionen $\delta$, die Rauigkeiten der Grenzschichten $\sigma$ und ebenfalls die Dicke der Polysterolschicht bestimmt werden. 
Es werden Rauigkeits-korrigierte Fresnelkoeffizienten aus \autoref{eqn:r_korrigiert} verwendet. 
Die so entstehende Fitfunktion ist dem Programmcode im Anhang \ref{sec:Anhang} zu enthemen. Es wird ein Fit an die korrigierten Messdaten mittels \texttt{scipy} \cite{scipy}
durchgeführt, wobei die Startwerte der freien Parameter auf zuvor berechnete und aus der Literatur bekannte Werte festgelegt werden. Ebenfalls wird der Wertebereich der Paramter 
um die zu erwartenden Größenordnungen eingeschränkt, um eine Konvergenz der Minimierung zu gewährleisten. Des Weiteren werden nur Messdaten zwischen 
$\qtyrange{0.3}{1.3}{\degree}$ berücksichtigt. Die ermittelte Ausgleichsfunktion und die Messdaten sind in \autoref{fig:Reflek3} dargestellt. 
\begin{figure}
  \centering
  \includegraphics[width = .7\textwidth]{Reflek3.pdf}
  \caption{Korrigierte Messdaten der Reflektivitäten und die ermittelte Ausgleichsfunktion des Parrattalgorithmus unter Berücksichtigung der Rauigkeitskorrektur.}
  \label{fig:Reflek3}
\end{figure}
Die freien Parameter der beiden Schichten ergeben sich zu 
\begin{align*}
  &\delta_\text{Poly} = \num{2.58 +- 1.77e-6} & &\delta_\text{Si} = \num{1.17 +- 0.09e-5} \\
  &\beta_\text{Poly} = \num{2.80 +- 1.34e-7} & &\delta_\text{Si} = \num{1.78 +- 131e-7} \\
  &\sigma_\text{Luft, Poly} = \num{9.14 +- 661e-9} & &\sigma_\text{Poly, Si} = \num{2.06 +- 8.22e-10} \\
  &d = \qty{9.31 +- 4.88e-8}{\metre}.
\end{align*}
Mit \autoref{eq:alpha_tot} können wieder die kritischen Winkel der beiden Materialien berechnet werden. Sie ergeben sich zu 
$\alpha_\text{c, Poly} = \qty{0.28 +- 0.01}{\degree}$ und $\alpha_\text{c, Si} = \qty{0.13 +- 0.04}{\degree}$.
