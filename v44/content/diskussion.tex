\section{Diskussion}
\label{sec:Diskussion}
Zuerst wurde die Halbwertsbreite und maximale Intensität des Röntgenstrahls bestimmt. Die dazu aufgenommenen Messdaten sind jedoch nicht perfekt symmetrisch und weisen 
besonders auf der rechten Seite der Gaußglocke einen welligen Verlauf auf. Dies hat zur Folge, dass die angepasste Gaußfunktion an einigen Stellen von der Kurve der 
Messdaten abweicht und so die bestimmten Werte der Intensität und der Halbwertsbreite mit einer gewissen Unsicherheit behaftet sind. Dies wirkt sich auch auf die späteren
Messergebnisse aus.\\
Über den erste durchgeführten Z-Scan konnte eine  Strahlbreite von $\qty{0.12}{\milli\metre}$ ermittelt werden. Da jedoch zwischen den beiden Plateubereichen der Intensität 
wenig Messwerte zur Verfügung stehen, ist auch dieser Wert mit einer hohen Unsicherheit behaftet. Auffällig ist auch, dass ein Peak im unteren Plateubereich registriert wurde,
der beispielsweise durch einen Kratzer auf dem Siliziumwafer erklärt werden könnte. 
Der experimentell bestimmte Geometriewinkel $\alpha_\text{G, exp} = \qty{0.44}{\degree}$ weicht um $\qty{29}{\percent}$ von dem mit der zuvor bestimmten Strahlbreite 
berechneten Wert von $\alpha_\text{G, Theorie} = \qty{0.34}{\degree}$ ab. Da jedoch der berechnete Wert von der zuvor bestimmten Strahlbreite abhängt und diese wie bereits 
erläutert von einer Unsicherheit behaftet ist, überrascht diese Abweichung nicht. Für die weiteren Berechnungen wurde deshalb der Wert 
$\alpha_\text{G, exp} = \qty{0.44}{\degree}$ verwendet. \\
Die Messdaten des Reflektivität-Scans zeigen eine gut erkennbare, oszillierende Struktur im Winkelbereich von $\qtyrange{0.3}{1.3}{\degree}$. Auch die in \autoref{fig:Reflek2}
eingezeichnete Fresnelreflektivität einer ideal glatten Oberfläche, stimmt für $\alpha > \alpha_\text{c}$ annähernd mit dem gemessenen Verlauf überein. Die Über die 
Kiessigoszillation bestimmte Schichtdicke des Polysterolfilms beträgt $d = \qty{8.89 +- 0.82e-8}{\metre}$.\\
Die über den Parrattalgorithmus Algorithmus angepasste Kurve stimmt ebenfalls grob mit der Messkurve überein. Bei feinerer Betrachtung fallen jedoch deutliche Abweichungen auf.
Um die Konvergenz des durchgeführten Fits zu erzielen, mussten die Startwerte und Wertebereiche der Parameter stark eingeschränkt werden, dennoch verbleiben einige Werte mit einer 
hohen statistischen Unsicherheit.
Die so bestimmten Dispersionen lauten  $\delta_\text{Poly, exp} = \num{2.58 +- 1.77e-6}$ und $\delta_\text{Si, exp} = \num{1.17 +- 0.09e-5}$. Die Literaturwerte sind
$\delta_\text{Poly, lit} = \num{3.5e-6}$ und $\delta_\text{Si, lit} = \num{7.6e-6}$ \cite{V44}. Dies bedeutet relative Abweichungen von $\qty{26}{\percent}$ und 
$\qty{85}{\percent}$ für Polysterol und Silizium respektive. Aus diesen Werten folgen die kritischen Winkel $\alpha_\text{c}$ der Materialien, die sich zu 
$\alpha_\text{c, exp}(\text{Poly}) =\qty{0.28 +- 0.01}{\degree}$ und $\alpha_\text{c, exp}(\text{Si}) = \qty{0.13 +- 0.04}{\degree}$ ergeben. Die Literaturwerte lauten 
$\alpha_\text{c, lit}(\text{Poly}) =\qty{0.153}{\degree}$ und $\alpha_\text{c, lit}(\text{Si}) =\qty{0.223}{\degree}$. Es folgen relative Abweichungen von $\qty{42}{\percent}$
für die Polysterol Werte und  $\qty{83}{\percent}$ für Silizium. Die durch den Parrattalgorithmus ermittelte Schichtdicke $d = \qty{9.31 +- 4.88e-8}{\metre}$ weicht um 
$\qty{5}{\percent}$ von dem zuvor bestimmten Wert ab. 
Die Rauigkeiten der beiden Grenzschichten wurden zu $\sigma_\text{Luft, Poly} = \num{9.14 +- 661e-9}$ $\sigma_\text{Poly, Si} = \num{2.06 +- 8.22e-10}$ bestimmt. \\
Allgemein sind die durch den Parrattalgorithmus bestimmten Messergebnisse von sehr großen Unsicherheiten behaftet.
Auf Grund der vielen freien Parameter und der komplex strukturierten Messkurve konnte keine gute Anpassung des Algorithmus gefunden werden.
Die Materialeigenschaften konnten nicht genau bestimmt werden. Lediglich die Schichtdicke konnte mit ausreichnder Präzision ermitttelt werden. 
