\section{Auswertung}
\label{sec:Auswertung}

\subsection{Dependence of the contrast on the polarisation angle}
At first the dependence of the contrast on the polarisation angle phi is analysed. The measurements listed in \autoref{tab:contrast} show the minimum and maximum intensity 
of the lasers interference for different polarisation angles $\phi$.
\begin{table}
  \centering
  \caption{Measurements for the polarisation angle dependence of the contrast $K$.}
  \label{tab:contrast}
  \begin{tabular}{c S[table-format = 1.2] S S S S S}
    \toprule
    {$\phi \mathbin{/} \unit{\degree}$} & {$I_\text{min1} \mathbin{/} \unit{\volt}$} & {$I_\text{max1} \mathbin{/} \unit{\volt}$} &%
    {$I_\text{min2} \mathbin{/} \unit{\volt}$} & {$I_\text{max2} \mathbin{/} \unit{\volt}$} &%
    {$I_\text{min3} \mathbin{/} \unit{\volt}$} & {$I_\text{max3} \mathbin{/} \unit{\volt}$} \\
    \midrule
      0 & 1.60 & 1.75 & 1.57 & 1.77 & 1.63 & 1.78 \\
     15 & 0.97 & 1.57 & 0.94 & 1.49 & 0.95 & 1.54 \\
     30 & 0.51 & 1.28 & 0.52 & 1.27 & 0.53 & 1.30 \\
     45 & 0.39 & 1.31 & 0.40 & 1.34 & 0.40 & 1.36 \\
     60 & 0.53 & 1.74 & 0.54 & 1.78 & 0.53 & 1.76 \\
     75 & 0.96 & 2.23 & 0.97 & 2.13 & 0.98 & 2.21 \\
     90 & 1.87 & 2.05 & 1.97 & 2.41 & 2.01 & 2.24 \\
    105 & 1.81 & 3.10 & 1.78 & 3.22 & 1.84 & 3.53 \\
    120 & 1.31 & 4.30 & 1.35 & 4.23 & 1.41 & 4.47 \\
    135 & 1.15 & 5.06 & 1.23 & 4.78 & 1.21 & 5.05 \\ 
    150 & 1.39 & 4.44 & 1.44 & 4.32 & 1.47 & 4.54 \\
    165 & 1.65 & 3.16 & 1.69 & 3.19 & 1.73 & 3.33 \\
    180 & 1.66 & 1.87 & 1.64 & 1.82 & 1.80 & 2.01 \\
    \bottomrule
  \end{tabular}
\end{table}
In order to compute the contrast, \autoref{eq:...} \textbf{EQREF HERE} is used to calculate the contrast 
$K$ for each measuremnt series. After that the average values and the standard deviations of the three measurements are calculated. The corresponding datapoints are shown in 
\autoref{fig:contrast}. 
\begin{figure}
  \centering
  \includegraphics[width = .8\textwidth]{contrast.pdf}
  \caption{Averaged measurements of the contrast $K$ against the polarisation angle $\phi$ and fit using \textit{scipy} \cite{scipy}.}
  \label{fig:contrast}
\end{figure}
The theory law of the angle dependence is given by \autoref{eq:...} \textbf{EQREF HERE}. Here, a function of the form 
\begin{equation*}
  K = 2K_0 \cdot \lvert \symup{sin}(\phi - \delta)\symup{cos}(\phi - \delta) \rvert
\end{equation*}
is fitted to the data points. The offset $\delta$ is used to compensate for deviations in the experimental setup. 
The fitparameters follow as 
\begin{align*}
  K_0 &= \num{0.57 +- 0.01} & \delta &= \qty{3.23 +- 0.58}{\degree}
\end{align*}
using the \textit{python} extension \textit{scipy} \cite{scipy} and only include statistical uncertainties. The resulting fit function is also displayed in \autoref{fig:contrast}.
For the following measurements the polarisation angle is set to $\qty{45}{\degree}$.

\subsection{Refraction index of glass}
For the determination of the refraction index of glass the double glass holder is placed in the two beams. The two glass panes are already tilted by an angle 
$\Theta_0 = \pm\qty{10}{\degree}$. 
Using \autoref{eq:...} \textbf{EQREF HERE} the number if maxima passing the center of the interference spectrum is given by
\begin{equation*}
  M = \frac{\symup{\Delta}\phi_+ + \symup{\Delta}\phi_-}{2\pi}
\end{equation*}
where $\symup{\Delta}\phi_\pm$ is the phase shift induced by the glass panes tilted by $\pm\qty{10}{\degree}$. 
This expression can be simplified to 
\begin{equation}
  M = \frac{2T}{\lambda} \cdot \frac{n-1}{n} \cdot \Theta_0 \theta
\end{equation}
where $\lambda = \qty{632.99}{\nano\metre}$ is the wavenlenght of the laser and $T = \qty{1}{\milli\metre}$ is the thickness of the glass panes.
The number of measured maxima is again averaged over the ten measurement series. The values are shown in \autoref{tab:n_glass}.
\begin{table}
  \centering
  \caption{Measurements of the maxima $M$ passing the center of the interference spectrum and corresponding tilt angle $\theta$.}
  \label{tab:n_glass}
  \begin{tabular}{ c c}
    \toprule
    {$\theta$} & {$\overline{M}$} \\
    \midrule
    {2} & {\num{ 6.0 +- 0.45}} \\
    {4} & {\num{12.3 +- 0.46}} \\
    {6} & {\num{19.2 +- 0.75}} \\
    {8} & {\num{25.3 +- 0.64}} \\
    \bottomrule
  \end{tabular}
\end{table}
The experimental value of the refraction index of glass follows from a linear fit to the datapoints.
The datapoints and the resulting fit are shown in \autoref{fig:n_glass}.
\begin{figure}
  \centering
  \includegraphics[width = .8\textwidth]{n_glass.pdf}
  \caption{Averaged measurements of the number of maxima $\overline{M}$ against the tilt angle $\theta$ and fit using \textit{scipy} \cite{scipy}.}
  \label{fig:n_glass}
\end{figure}
The resulting value is $n_\text{glass} = \num{1.484 +- 0.008}$.

\subsection{Refraction index of air}
To determine the refraction index of air, the measurements listed in \autoref{tab:n_air} are used.
\begin{table}
  \centering
  \caption{Measurements of the maxima $M$ passing the center of the interference spectrum and pressure $p$ in the air chamber.}
  \label{tab:n_air}
  \begin{tabular}{c c c c c c}
    \toprule
    {$p \mathbin{/} \unit{\milli\bar}$} & {$M_1$} & {$M_2$} & {$M_3$} & {$M_4$} & {$M_5$}\\
    \midrule
    {  8} & { 0} & { 0} & { 0} & { 0} & { 0} \\
    { 50} & { 2} & { 2} & { 2} & { 2} & { 2} \\
    {100} & { 4} & { 4} & { 4} & { 4} & { 4} \\
    {150} & { 6} & { 6} & { 6} & { 7} & { 6} \\
    {200} & { 8} & { 9} & { 9} & { 9} & { 9} \\
    {250} & {10} & {11} & {11} & {11} & {11} \\
    {300} & {13} & {13} & {13} & {13} & {13} \\
    {350} & {15} & {15} & {15} & {15} & {15} \\
    {400} & {17} & {17} & {17} & {17} & {17} \\
    {450} & {19} & {19} & {19} & {19} & {19} \\
    {500} & {21} & {21} & {21} & {21} & {21} \\
    {550} & {23} & {23} & {23} & {23} & {23} \\
    {600} & {25} & {25} & {25} & {25} & {25} \\
    {650} & {27} & {28} & {28} & {27} & {27} \\
    {700} & {29} & {30} & {30} & {30} & {30} \\
    {750} & {32} & {32} & {32} & {32} & {32} \\
    {800} & {34} & {34} & {34} & {34} & {34} \\
    {850} & {36} & {36} & {36} & {36} & {36} \\
    {900} & {38} & {38} & {38} & {38} & {38} \\
    {950} & {40} & {40} & {40} & {40} & {40} \\
    {981} & {41} & {41} & {41} & {41} & {41} \\
    \bottomrule
  \end{tabular}
\end{table}
Again the average over the five measurement series is calculated.

\begin{figure}
  \centering
  \includegraphics{n_air.pdf}
  \caption{Plot.}
  \label{fig:n_air}
\end{figure}
