\section{Analysis}
\label{sec:Auswertung}

\subsection{Dependence of the contrast on the polarisation angle}
At first the dependence of the contrast on the polarisation angle $\phi$ is analysed. The measurements listed in \autoref{tab:contrast} show the minimum and maximum intensity 
of the lasers interference for different polarisation angles $\phi$.
\begin{table}
  \centering
  \caption{Measurements for the polarisation angle dependence of the contrast $K$.}
  \label{tab:contrast}
  \begin{tabular}{c S[table-format = 1.2] S S S S S}
    \toprule
    {$\phi \mathbin{/} \unit{\degree}$} & {$I_\text{min1} \mathbin{/} \unit{\volt}$} & {$I_\text{max1} \mathbin{/} \unit{\volt}$} &%
    {$I_\text{min2} \mathbin{/} \unit{\volt}$} & {$I_\text{max2} \mathbin{/} \unit{\volt}$} &%
    {$I_\text{min3} \mathbin{/} \unit{\volt}$} & {$I_\text{max3} \mathbin{/} \unit{\volt}$} \\
    \midrule
      0 & 1.60 & 1.75 & 1.57 & 1.77 & 1.63 & 1.78 \\
     15 & 0.97 & 1.57 & 0.94 & 1.49 & 0.95 & 1.54 \\
     30 & 0.51 & 1.28 & 0.52 & 1.27 & 0.53 & 1.30 \\
     45 & 0.39 & 1.31 & 0.40 & 1.34 & 0.40 & 1.36 \\
     60 & 0.53 & 1.74 & 0.54 & 1.78 & 0.53 & 1.76 \\
     75 & 0.96 & 2.23 & 0.97 & 2.13 & 0.98 & 2.21 \\
     90 & 1.87 & 2.05 & 1.97 & 2.41 & 2.01 & 2.24 \\
    105 & 1.81 & 3.10 & 1.78 & 3.22 & 1.84 & 3.53 \\
    120 & 1.31 & 4.30 & 1.35 & 4.23 & 1.41 & 4.47 \\
    135 & 1.15 & 5.06 & 1.23 & 4.78 & 1.21 & 5.05 \\ 
    150 & 1.39 & 4.44 & 1.44 & 4.32 & 1.47 & 4.54 \\
    165 & 1.65 & 3.16 & 1.69 & 3.19 & 1.73 & 3.33 \\
    180 & 1.66 & 1.87 & 1.64 & 1.82 & 1.80 & 2.01 \\
    \bottomrule
  \end{tabular}
\end{table}

In order to compute the contrast $K$, \autoref{eq:Contrast} is applied for each measurement series. After that, the average values and the standard deviations of the three measurements are calculated. The corresponding datapoints are shown in 
\autoref{fig:contrast}. 
\begin{figure}
  \centering
  \includegraphics[width = .8\textwidth]{contrast.pdf}
  \caption{Averaged measurements of the contrast $K$ against the polarisation angle $\phi$ and fit using \textit{scipy} \cite{scipy}.}
  \label{fig:contrast}
\end{figure}
The theory law of the angle dependence is given by \autoref{eq:contrast_theo_func}. Here, a function of the form 
\begin{equation*}
  K = 2K_0 \cdot \lvert \symup{sin}(\phi - \delta)\symup{cos}(\phi - \delta) \rvert
\end{equation*}
is fitted to the data points. The offset $\delta$ is used to compensate for deviations in the experimental setup. 
The fitparameters follow as 
\begin{align*}
  K_0 &= \num{0.57 +- 0.01} & \delta &= \qty{3.23 +- 0.58}{\degree}
\end{align*}
using the \textit{python} extension \textit{scipy} \cite{scipy}. The resulting fit function is also displayed in \autoref{fig:contrast}.
For the following measurements the polarisation angle is set to $\qty{45}{\degree}$.

\subsection{Refractive index of glass}
For the determination of the refractive index of glass the double glass holder is placed in the two beams. The two glass panes are already tilted by an angle 
$\Theta_0 = \pm\qty{10}{\degree}$. 
Using the equations \eqref{eq:M} and \eqref{eq:delta_glass} the number of maxima passing the center of the interference spectrum is given by
\begin{equation*}
  M = \frac{\symup{\Delta}\phi_+ + \symup{\Delta}\phi_-}{2\pi}
\end{equation*}
where $\symup{\Delta}\phi_\pm$ is the phase shift induced by the glass panes tilted by $\pm\qty{10}{\degree}$. 
This expression can be simplified to 
\begin{equation}
  \label{eq:M_n_glass}
  M = \frac{2T}{\lambda} \cdot \frac{n-1}{n} \cdot \Theta_0 \theta
\end{equation}
where $\lambda = \qty{632.99}{\nano\metre}$ is the wavenlength of the laser and $T = \qty{1}{\milli\metre}$ is the thickness of the glass panes.
The number of measured maxima is again averaged over the ten measurement series. The values are shown in \autoref{tab:n_glass}.
\begin{table}
  \centering
  \aboverulesep=0ex % Solution part 1 of 3
  \belowrulesep=0ex % Solution part 1 of 3
  \caption{Measurements of the maxima $M$ passing the center of the interference spectrum and tilt angle $\theta$.}
  \label{tab:n_glass}
  \begin{tabular}{c | c c c c c c c c c c | c}
    \toprule
    {$\theta$} & {$M_1$} & {$M_2$} & {$M_3$} & {$M_4$} & {$M_5$} & {$M_6$} & {$M_7$} & {$M_8$} & {$M_9$} & {$M_{10}$}  & {$\overline{M}$} \\
    \midrule
    \rule{0pt}{1.1EM}
    {2} &  6 &  6 &  6 &  6 &  5 &  6 &  6 &  6 &  7 &  6 & {\num{ 6.0 +- 0.45}} \\
    {4} & 12 & 13 & 12 & 13 & 12 & 12 & 12 & 12 & 13 & 12 & {\num{12.3 +- 0.46}} \\
    {6} & 19 & 20 & 19 & 20 & 18 & 19 & 18 & 20 & 20 & 19 & {\num{19.2 +- 0.75}} \\
    {8} & 25 & 26 & 25 & 26 & 24 & 25 & 25 & 26 & 26 & 25 & {\num{25.3 +- 0.64}} \\
    \bottomrule
  \end{tabular}
\end{table}
The experimental value of the refractive index of glass follows from a linear fit of \autoref{eq:M_n_glass} to the datapoints.
The datapoints and the fit are shown in \autoref{fig:n_glass}.
\begin{figure}
  \centering
  \includegraphics[width = .8\textwidth]{n_glass.pdf}
  \caption{Averaged measurements of the number of maxima $\overline{M}$ against the tilt angle $\theta$ and fit using \textit{scipy} \cite{scipy}.}
  \label{fig:n_glass}
\end{figure}
The resulting value is $n_\text{glass} = \num{1.484 +- 0.008}$.

\subsection{Refractive index of air}
To determine the refractive index of air, the measurements listed in \autoref{tab:n_air} are used.
Again the average over the five measurement series is calculated. Using \autoref{eq:n_gas} and the length $L = \qty{100 +- 0.1}{\milli\metre}$ of the air chamber 
the corresponding refractive indices of air can be calculated at each 
pressure value. The resulting refractive indices are also listed in \autoref{tab:n_air} and are shown against the pressure in \autoref{fig:n_air}.
\begin{table}
  \centering
  \aboverulesep=0ex % Solution part 1 of 3
  \belowrulesep=0ex % Solution part 1 of 3
  \caption{Measurements of the maxima $M$ passing the center of the interference spectrum and pressure $p$ in the air chamber. 
  The refractive indices are displayed for the averaged value for each pressure.}
  \label{tab:n_air}
  \begin{tabular}{c | c c c c c | c}
    \toprule
    {$p \mathbin{/} \unit{\milli\bar}$} & {$M_1$} & {$M_2$} & {$M_3$} & {$M_4$} & {$M_5$} & {($\overline{n} -1) \mathbin{/} 10^{-5}$}\\
    \midrule
    \rule{0pt}{1.1EM}
    {  8} & { 0} & { 0} & { 0} & { 0} & { 0} & \num{0 +-0}\\
    { 50} & { 2} & { 2} & { 2} & { 2} & { 2} & \num{1.27+-0.00}\\
    {100} & { 4} & { 4} & { 4} & { 4} & { 4} & \num{2.53+-0.00}\\
    {150} & { 6} & { 6} & { 6} & { 7} & { 6} & \num{3.92+-0.25}\\
    {200} & { 8} & { 9} & { 9} & { 9} & { 9} & \num{5.57+-0.25}\\
    {250} & {10} & {11} & {11} & {11} & {11} & \num{6.84+-0.25}\\
    {300} & {13} & {13} & {13} & {13} & {13} & \num{8.23+-0.01}\\
    {350} & {15} & {15} & {15} & {15} & {15} & \num{9.49+-0.01}\\
    {400} & {17} & {17} & {17} & {17} & {17} & \num{10.76+-0.01}\\
    {450} & {19} & {19} & {19} & {19} & {19} & \num{12.03+-0.01}\\
    {500} & {21} & {21} & {21} & {21} & {21} & \num{13.29+-0.01}\\
    {550} & {23} & {23} & {23} & {23} & {23} & \num{14.56+-0.01}\\
    {600} & {25} & {25} & {25} & {25} & {25} & \num{15.82+-0.02}\\
    {650} & {27} & {28} & {28} & {27} & {27} & \num{17.34+-0.31}\\
    {700} & {29} & {30} & {30} & {30} & {30} & \num{18.86+-0.25}\\
    {750} & {32} & {32} & {32} & {32} & {32} & \num{20.26+-0.02}\\
    {800} & {34} & {34} & {34} & {34} & {34} & \num{21.52+-0.02}\\
    {850} & {36} & {36} & {36} & {36} & {36} & \num{22.79+-0.02}\\
    {900} & {38} & {38} & {38} & {38} & {38} & \num{24.05+-0.02}\\
    {950} & {40} & {40} & {40} & {40} & {40} & \num{25.32+-0.03}\\
    {981} & {41} & {41} & {41} & {41} & {41} & \num{25.95+-0.03}\\
    \bottomrule
  \end{tabular}
\end{table}
From these values, the refractive index of air at standard atmosphere ($T = \qty{15}{\degreeCelsius}$, $p = \qty{1013}{\hecto\pascal}$) can be optained. The Lorentz-Lorenz law 
(\autoref{eq:LLL}) can be approximated for $n \approx 1$ as 
\begin{equation*}
  n = \frac{3}{2}\frac{Ap}{RT} + 1.
\end{equation*}
Using this approach, the experimentally determined values of the refractive index in \autoref{fig:n_air} are fitted with a linear function 
\begin{equation*}
  n(p, T = T_0) = \frac{3}{2}\frac{p}{RT_0} \cdot a + b 
\end{equation*}
where $a$ and $b$ are the free parameters of the fit and $T_0 = \qty{22.2}{\degreeCelsius} = \qty{295.35}{\kelvin}$ is the measured room temperature.
The fit parameters determined using \text{scipy} \cite{scipy} follow as 
\begin{align*}
  a &= \num{4.38 +- 0.02e-4} & b &= 1 + \num{3 +- 66e-8}.
\end{align*}
The experimental value of the refractive index of air at standard atmosphere then reads $n_\text{exp} = 1 + \num{27.05+-0.13e-5}$.

\begin{figure}
  \centering
  \includegraphics[width = .8\textwidth]{n_air.pdf}
  \caption{Calculated refractive index of air against measured pressure and linear fit using \textit{scipy} \cite{scipy}.}
  \label{fig:n_air}
\end{figure}
