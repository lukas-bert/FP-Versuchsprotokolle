\section{Motivation}
\label{sec:Motivation}
In this experiment we aim at determining the refraction index of air and glass by using the priciples of interferometry. A Sagnac interferometer is used to achieve
effects of interference.

\section{Theory}
\label{sec:Theory}
When two wavefronts meet, interference phenomena can occur under certain conditions. Interference means that the waves add up according to the superposition principle. This can result in
intensity maxima and minima.
In the following, the physics behind interference and the connection to other physical properties like the refraction index is examined based on the example of the Sagnac interferometer.

\subsection{Coherence}
\label{sec:Coherence}
Two waves can interfere with other when they are coherent, meaning a constant phase relation is given. It is differentiated between temporal and spacial coherence. For the temporal
coherence, the phase relation stays the same for an infinet time. Spacial coherence describes the constant phase relation regarding the spacial direction of propagation.

In reality, the will be hardly any waves that are perfectly coherent. Nevertheless, a coherence length can be identified as the distance between waves under which the waves are sufficiently
coherent. The degree of coherence $\gamma_{12}$ is given by
\begin{equation*}
    \gamma_{12}(\tau)= \frac{\langle E_1(t+\tau)E^{*}_2(t) r \rangle}{\sqrt{\langle|E_1|^2\rangle\langle|E_2|^2\rangle}}. CHECK FORMULA
\end{equation*}
It becomes clear that the lower $|\gamma_{12}|$ the less the light is coheren with $0≤|\gamma_{12}|≤1$.

\subsection{Polarization}
\label{sec:Polarization}
Another important property of light is its polarization. The polarization of a light beam describes the direction in which the electric or magnetic field oscillates. Normal sunlight 
is unpolarized and thus has no distinguished oscillation direction of the electric or magnetic field. There are several ways to polarise light beams, for example polarization filters that only
let light pass that is linearly polarized under a certain angle.

\subsection{Sagnac Interferometer}
\label{sec:Sagnac_Interferometer}
