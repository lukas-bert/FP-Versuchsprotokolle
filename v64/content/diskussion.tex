\section{Discussion}
\label{sec:Diskussion}

The dependence of the contrast on the polarisation angle showed good agreement between the expected theory curve and the observed data. An angle offset of 
$\delta = \qty{3.23 +- 0.58}{\degree}$ was observed compared to the theory which could possibly be explained by a constant offset on the scale of the polariser or other 
deviations in the experimental setup. The maximum contrast reaches a value of $K_0 = \num{0.57 +- 0.01}$ where the ideal contrast would be $K = 1$. This could hint to a 
suboptimal alignment. \\
The refractive index of glass was determined to be $n_\text{glass, exp} = \num{1.484 +- 0.008}$. In the literature the value is given by $n_\text{glass, theory} \approx \num{1.515}$
\cite{Refractive_Index}, but many different glass types exist with different refractive indices.
The relative deviation of the experimental value is $\symup{\Delta}_\text{rel} (n_\text{glass}) = \qty{2}{\percent}$. Considering the general experimental uncertainties and the fact, that the exact 
composition of the glass panes used in the experiment is unknown, the refractive index of glass was determined with sufficient precision.\\
Lastly, the refractive index of air at standard atmosphere ($\qty{15}{\degreeCelsius}$, $\qty{1013}{\hecto\pascal}$) was determined as $n_\text{air, exp} = \num{1.0002705+-0.0000013}$.
The literature value is $n_\text{air, theory} = 1.00027653$~\cite{Refractive_Index} which implies a deviation of $\symup{\Delta}_\text{rel} (n_\text{air}) < \qty{0.001}{\percent}$.
The small deviation could e.g. be caused by humidity in the air which was neglected for the theory value. Therefore, the refractive index of air was determined precisely.
All in all, the refractive indices were determined successfully, but a more careful alignment of the Sagnac-interferometer could lead to a higher contrast which may increase precision.
