\section{Diskussion}
\label{sec:Diskussion}
Die experiementell bestimmten Leerlaufverstärkungen werden in \autoref{tab:vlg_V} mit den theoretischen verglichen. Es fällt auf, 
dass für $V_{\mathrm{theo}}=15$ der experiementelle Wert mit dem Unsicherheitsbereich vereinbar ist. Für diese Messung konnte über einen 
großen Plateaubereich gemittelt werden (siehe \autoref{fig:plots_inv}), was für die anderen beiden Verstärkungen nicht der Fall ist. Daher
liegt die Vermutung nahe, dass mit einem größeren Datensatz auch die anderen Verstärkungen näher am Theoriewert lägen.
\begin{table}
    \centering
    \caption{Vergleich der theoretischen und experiementell bestimmten Leerlaufverstärkungen}
    \label{tab:vgl_V}
    \begin{tabular}{S S S}
      \toprule
      {$V_{\mathrm{theo}}$} & {$V_{\mathrm{i}}$} & {rel. Abweichung} \\
      \midrule
      100 & \num{92+-5}     & \qty{8}{\percent} \\
       15 & \num{15.1+-0.4} & \qty{0,7}{\percent} \\
      179 & \num{129+-11}   & \qty{27,9}{\percent} \\ 
      \bottomrule
    \end{tabular}
  \end{table}

  Die bestimmten Grenzfrequenzen und Bandbreiten hängen auch von den bestimmten Leerlaufverstärkungen ab, sodass sich die Unsicherheiten 
  hier fortpflanzen.

  Die Funktionsweise der Integratorschaltung kann in diesem Versuch bestätigt werden. In \autoref{fig:scope_int} ist deutlich die
  integrierende Wirkung der Schaltung visualisiert. In dem zur Bestimmung der Zeitkonstanten verwendete Plot \autoref{fig:integrator_exp} ist zu
  sehen, dass der erwartete Theoriewert und der experiementelle Wert voneinander abweichen. Im Detail liegen die beiden Werte
  der beiden Werte
  \begin{align*}
    \tau_{\mathrm{int, theo}} &=\qty{1}{\milli\second} \\
    \tau_{\mathrm{int, exp}} &= \qty{6.40+-0.21}{\milli\second} \\
  \end{align*}
 soweit auseinander, dass diese nicht durch statistische Unsicherheiten erklärt werden können. Es ist denkbar, dass ein Bauteil fehlerhaft ist
 oder ein falsches Bauteil in die Schaltung eingesetzt wurde.

 Bei der Differentiatorschaltung kann keine differnezierende Wirkung festgestellt werden. Stattdessen ist zu beobachten, dass die Schaltung
 genau wie die Integratorschaltung zuvor funktioniert. Auch der Fit in \autoref{fig:differenzierer_exp} liefert eine Steigung von $m\approx -1$, 
 was charakteristisch für eine Integratorschaltung ist. Es ist daher nicht auszuschließen, dass fahrlässiger Weise zweimal dieselbe Schaltung
 aufgebaut wurde.

 Für den Schmitt-Trigger wird die theoretische Schwellenspannung von $U_{\mathrm{kipp, theo}} = \qty{1,5}{\volt}$ experiementell bestätigt.
 Minimale Abweichung lassen sich durch Innen- oder Kabelwiderstände erklären.

 Für die erste Generatorschaltung weichen die bestimmten Werte der Frequenz und Spannung stark von den theoretischen ab. Es ist auch hier 
 nicht unvorstellbar, dass falsche Bauteile eingesetzt wurden.
 Die Periodendauer der Generatorschaltung mit variierender Amplitude wird zu
\begin{equation*}
  T_{\mathrm{exp}} = \qty{5.56+-0.32}{\milli\second}
\end{equation*}
bestimmt, was einer Abweichung von \qty{12,9}{\percent} zu der theoretischen von
\begin{equation*}
    T_{\mathrm{theo}} = \qty{6,28}{\milli\second}
\end{equation*}
entspricht. Möglich ist hier, dass die bestimmten lokalen Maxima nicht exakt genug sind. Hierbei ist darüber hinaus verwunderlich, dass
die eingekoppelte Rechteckspannung nicht für alle Frequenzen sichtbar bleibt (vgl. \autoref{fig:scope_generator2}). Etwaige Rückkopplungen
könnten ebenfalls das Ergbniss verfälschen.

%  \begin{align*}
%     \nu_\text{a} &=\qty{2,5}{\kilo\hertz} \\
%     U_0 &=\qty{1,5}{\volt} \\
%   \end{align*}
%   \begin{align*}
%     \nu_{\mathrm{a, exp}} &= \qty{4,678}{\kilo\hertz} \\
%     U_{0, \mathrm{exp}} &= \qty{1,09}{\volt}.
%   \end{align*}