\section{Motivation}
\label{sec:Motivation}
The aim of this experiment is to determine the specific heat capacity $C$ of copper and its temperature dependence at low temperatures and room temperature.
Therefore, the specific heat capacity at constant pressure $C_{\symup{p}}$ is measured and translated to the specific heat capacity at constant volume $C_{\symup{V}}$.
Different models to describe the temperature dependency are compared and experimentally tested. These are the classical model, the Einstein model and the Debye model.
Furthermore, the \textit{Debye temperature} $\Theta_{\symup{D}}$ is
derived from the experiment's results and compared with the model's theoretical expectations.

\section{Theory}
\label{sec:Theory}
The heat capacity of a material describes the amount of heat that is needed to increase the temperature of a certain amount of the material by $\qty{1}{\kelvin}$.
In general, it can be calculated as
\begin{equation*}
    C = \frac{\delta Q}{\delta T}
\end{equation*}
where $\delta Q$ is the input heat and $\delta T$ the change in temperature.
Most often the molar heat capacity $c^\text{m}$ is used, but also the heat capacity per mass $c^\text{mass}$ or the heat capacity per volume $c^\text{vol}$ can be applied.
As mentioned before, it is differentiated between heat capacity at constant volume $C_{\symup{V}}$ and constant pressure $C_{\symup{p}}$.
For $C_{\symup{V}}$ the equation
\begin{equation}
    \label{eqn:CV}
    C_{\symup{V}} = \frac{\delta U}{\delta T} \biggr\rvert_{\mathbf{V}}
\end{equation}
can be used to calculate the heat capacity, where $U$ is the internal energy of the system. Experimentally, it is often easier to measure the heat capacity
\begin{equation}
    C_{\symup{p}} = \frac{\delta Q}{\delta T} \biggr\rvert_{\mathbf{p}}
    \label{eq:Cp}
\end{equation}
at constant pressure because most materials expand when heated.
The deviation of $C_{\symup{p}}$ and $C_{\symup{V}}$ can be corrected using
\begin{equation}
    \label{eqn:Cp_CV}
    C_{\symup{p}} - C_{\symup{V}} = 9 T V \alpha^2_V B
\end{equation}
where $\alpha_{\symup{V}}$ is the volumetric expansion coefficient and $B$ the so-called bulk module.

\subsection{Classical theory of heat capacity}
\label{subsec:Classical}

In classical thermodynamics, the equipartition theorem states that the thermal energy of a solid is evenly distributed on its degrees of freedom and every degree of
freedom corresponds to $\frac{1}{2}k_{\symup{B}} T$ of kinetic and potential energy, respectively. Here, $k_{\symup{B}} $ is the Boltzmann constant.
Assuming a crystal of $N$ unit cells (1 atom per cell), this results in a total internal energy of
\begin{equation*}
    U = U^\text{eq} + 3N \cdot 2 \frac{1}{2}k_{\symup{B}}  T = U^\text{eq} + 3Nk_{\symup{B}}  T.
\end{equation*}
Using \autoref{eqn:CV} the heat capacity at constant volume reads
\begin{equation*}
    C_{\symup{V}} = 3Nk_{\symup{B}} .
\end{equation*}
For the molar heat capacity, the \textit{Dulong-Petit} law
\begin{equation}
    c^{\symup{m}}_{\symup{V}} = 3 R
\end{equation}
can be derived. Here $R = N_{\symup{A}} k_{\symup{B}}$ is the gas constant and $N_{\symup{A}}$ the Avogadro constant (number of atoms in $\qty{1}{\mol}$).
The classical approach leads to a heat capacity that is material- and temperature independent. Experimental results however, show that at low temperatures, the heat capacity is
proportional to $T^3$ and only approaches the classical value for $C_{\symup{V}}$ at higher temperatures. Also, a dependence on the material can be noticed in experiments.
Quantum mechanical effects have to be taken into account to explain this behaviour.

\subsection{The Einstein model}
\label{subsec:Einstein}
The Einstein approximation was the first theory to describe quantum dynamic effects on heat capacity.
The approximation assumes the same frequency $\omega_\text{E}$ for all $3N$ oscillation modes of the system. Only energies of whole multiples of
$\hbar \omega$ can be absorbed or emitted. With this approach
\begin{equation*}
    \langle U \rangle = 3N \hbar \omega_\text{E} \left(\frac{1}{2} + \frac{1}{\mathrm{e}^{\hbar \omega_\text{E}/k_{\symup{B}}T}}\right)
\end{equation*}
follows for the internal energy using the Bose-Einstein statistics.
The heat capacity will be
\begin{equation}
    \label{eqn:CV_Einstein}
    C^\text{E}_{\symup{V}} = 3Nk_{\symup{B}} \left(\frac{\Theta_\text{E}}{T}\right)^2 \frac{\mathrm{e}^{\Theta_\text{E}/T}}{\left(\mathrm{e}^{\Theta_\text{E}/T} -1 \right)^2},
\end{equation}
where $\Theta_\text{E} = \sfrac{\hbar \omega_\text{E}}{k_{\symup{B}}}$ is the Einstein temperature.
The approximation for low and high temperatures
\begin{equation*}
    C^\text{E}_{\symup{V}} =
    \begin{cases}
        3Nk_{\symup{B}}\left(\frac{\Theta_\text{E}}{T}\right)^2 \mathrm{e}^{-\Theta_\text{E}/T}, & T \ll \Theta_\text{E}\\
        3Nk_{\symup{B}}, & T \gg \Theta_\text{E}\\
    \end{cases}
\end{equation*}
yields the correct result for high temperatures but does not follow the experimental dependency for the low temperature case.
This is caused by the assumption that all atoms have the same frequency.

\subsection{The Debye model}
A better approximation is given by the Debye model, which introduces two fundamental assumptions: 1. All phonon branches are approximated by three (acoustic) branches with
a linear dispersion relation $\omega_i = v_i q$. 2. The summation over all possible wave vectors $q$ is replaced by an integration over the first Brillouin zone that can further
be simplified to a spherical integration with radius $q_\text{D}$. Because of the requirement that $N$ different states (per branch) have to be included in the sphere and using a
volume of $(2\symup{\pi}/L)^3$ per state, the wave vector reads $q_\text{D} = \left(6\symup{\pi}^2 \frac{N}{V}\right)^{\frac{1}{3}}$.
The Debye Frequency is $\omega_{\text{D}, i} = q_\text{D} v_i$ where $v_i$ is the speed of sound of the $i$-th branch.
The density of states can be written as
\begin{equation*}
    D_i(\omega) = \frac{V}{2\symup{\pi}^2}\frac{\omega^2}{v^3_i}.
\end{equation*}
Now the internal energy
\begin{equation*}
    U = \int^{\omega_\text{D}}_{0} \frac{D(\omega)}{\mathrm{exp}(\hbar\omega / k_{\symup{B}} T) -1} \,\symup{d}\omega
\end{equation*}
can be calculated.
Using the mean speed of sound $v_\text{S}$ of the three phonon branches and substituting $x = \hbar v_\text{S} q /k_{\symup{B}} T$, the heat capacity
\begin{equation}
    \label{eqn:CV_Debye}
    C^\text{D}_{\symup{V}} = 9N k_{\symup{B}} \left(\frac{T}{\Theta_\text{D}}\right)^3 \int^{\Theta_\text{D}/T}_{0} \frac{x^4 \mathrm{e}^x}{(\symup{e}^{-x} -1)^2} \, \symup{d}x
\end{equation}
can be derived.
This can be approximated as
\begin{equation*}
    C^\text{D}_{\symup{V}} =
    \begin{cases}
        \frac{12\symup{\pi}^4}{5}Nk_{\symup{B}} \left(\frac{T}{\Theta_\text{D}}\right)^3, & T \ll \Theta_\text{D}\\
        3Nk_{\symup{B}}, & T \gg \Theta_\text{D}\\
    \end{cases}
\end{equation*}
for low and high temperatures respectively. This result includes the classical value for $C_{\symup{V}}$ at high temperatures, as well as the experimental observation of
a $T^3$ dependency for low temperatures.
