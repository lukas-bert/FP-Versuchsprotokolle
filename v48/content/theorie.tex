\section{Zielsetzung}
In diesem Versuch werden die Eigenschaften eines Ionenkristalls (Strontium dotiertes Kaliumbromid) mithilfe der Ionen-Thermostrom Methode
(\textit{ITC: ionic thermocurrents}~\cite{PhysRev.148.816}) untersucht.
Dabei wird die temperaturabhängige Relaxationszeit der Dipole des Ionenkristall vermessen und so die charakteristische Relaxationszeit $\tau_0$, sowie die Aktivierungsenergie $W$
der Dipole bestimmt.

\section{Theorie}
\label{sec:Theorie}
Ionenkristalle sind Kristalle, bei denen der Beitrag der ionischen Bindung überwiegt. Die positiv geladenen Kationen (hier: \ce{K^+}) gehen dabei eine Bindung mit den negativ
geladenen Anionen (hier: \ce{Br^-}) ein und bilden ein kubisches Kristallgitter. In der Realität ist dieses Gitter jedoch nicht perfekt und es kommt zu Punktdefekten (Störstellen)
an denen beispielsweise ein Gitterplatz nicht besetzt ist. Leerstellen können sich wie Ladungsträger im Kristall bewegen oder von anderen Atomen besetzt werden.
Diese Eigenschaft wird sich bei der Dotierung von Kristallen zu Nutze gemacht, indem (geladene) Fremdatome in diese Störstellen eingebracht werden. 
Bei der Dotierung von Kaliumbromid mit Strontium nehmen zweifach positiv geladenen Strontiumatome eine Leerstelle ein. Damit lokal Ladungsneutralität gewährleistet ist,
wandert ein benachbartes Kaliumatom zu einer anderen Leerstelle oder an die Oberfläche des Kristalls.
Es entstehen elektrische Dipole, die sich innerhalb des Kirstallgitters orientieren können.

\subsection{Elektrische Dipole in Ionenkristallen}
Durch die unterschiedlich geladenen Atome bildet sich ein Dipolmoment
\begin{equation*}
    \vec{P} = \sum_i \vec{p}_i = \sum_i q \cdot \vec{r}_i
\end{equation*}
wobei $q$ die Ladung der Dipole und $r_i$ der Abstand ist. Ohne äußere Einflüsse sind die einzelnen Dipole des Kristalls zufällig ausgerichtet und das Gesamtdipolmoment verschwindet.
Wird jedoch ein elektrisches Feld $\vec{E}$ angelegt, kann die potentielle Energie 
\begin{equation*}
    E_\text{pot} = -\vec{p} \cdot \vec{E}
\end{equation*}     
der Dipole minimiert werden, indem sich die Dipole parallel zu den Feldlinien des elektrischen Feldes ausrichten. Da dazu die Leerstellen des Kirstalls ihre Position ändern müssen 
(Leerstellendiffusion) ist eine materialspezifische Aktivierungsenergie $W$ von nöten, um die Coulombbarriere des Gitterpontetials zu überwinden. 
Ist genügend Energie durch thermische Anregung gegeben, richten sich die Dipole entlang des $\vec{E}$-Feldes aus. Durch Abschalten des elektrischen Feldes relaxieren die Dipole 
wieder in eine zufällige Verteilung. Die Eneergieverteilung des Kristalls folgt dabei der Boltzmann-Statistik, weshalb sich eine temperaturabhängige Relaxationszeit 
\begin{equation}
    \label{eq:tau}
    \tau(T) = \tau_0 \mathrm{e}^{W/k_\text{B}T}
\end{equation}
angeben lässt. Die materialspezifische charakteristische Relaxationszeit des Kristalls ist $\tau_0 = \tau(\infty)$. 
Bei der Ionen-Thermostrom Methode wird der Kristall bei einer (hohen) Temperatur $T_1$ ausreichend lang durch ein elektrisches Feld $E$ polarisiert. 
Anschließend wird der Kristall auf eine Temperatur $T_0$ 
herunter gekühlt, sodass die durch \autoref{eq:tau} gegebene Relaxationszeit sehr lang ist. Durch Erwärmen des Kristalls relaxieren mehr und mehr Dipole, 
wodurch ein Strom an den senkrecht zum
$\vec{E}$-Feld liegenden Flächen des Kristalls messbar ist. Anhand des Temperaturverlaufs des Stroms lassen sich $\tau_0$ und $W$ bestimmen. 
Dazu werden zwei Herleitungen eines Zusammenhangs zwischen den gesuchten Größen und dem Strom betrachtet \cite{Fuller1972}.

\subsection{Herleitung über Polarisationsstrom}
\label{sec:Herleitung_Polarisationsstrom}
In dieser Herleitung ist der Ansatz des Polarisationsstroms durch die Überlegung 
\begin{equation}
    \label{eq:Ansatz}
    i(T) = \text{(\diameter Polarisierung)} \cdot \text{(\# beteiligte Dipole)} \cdot \text{(Rate der Dipolrelaxation)}
\end{equation} 
gegeben.
Die durchschnittliche Polarisierung ist dabei über die Debye-Polarisierung
\begin{equation*}
    \label{eq:debye}
    P(T) = \frac{p^2 E}{3k_\text{B} T}
\end{equation*}
gegeben, wobei $p$ das Dipolmoment eines einzelnen Dipols ist. Die Anzahl der bei der Temperatur $T$ beitragenden Dipole wird über die Relaxationsgleichung 
\begin{equation*}
\frac{\symup{d}N}{\symup{d}T} = -\frac{1}{\tau(T)} \cdot N
\end{equation*}
beschrieben und ist bei konstanter Heizrate $b$ über die Lösung 
\begin{align*}
N &=  N_0 \text{exp}\left(-\int_{t_0}^t \frac{1}{\tau(T)} \symup{d}t'\right) \\
&= N_0 \text{exp}\left( -\int_{T_0}^T (b\tau_0)^{-1} \symup{e}^{-W/k_\text{B}T'} \symup{d}T' \right)
\end{align*}
gegeben, wobei $N_0$ die Zahl der ursprünglich ausgerichteten Dipole bei $T_0$ ist. 
Die Rate der Dipolrelaxation ist das Inverse der Relaxationszeit (\autoref{eq:tau}).
Damit folgt aus \autoref{eq:Ansatz} für den Strom 
\begin{equation}
    \label{eq:i_T}
    i(T) = \frac{N_0 p^2 E}{3k_\text{B}T_1 \tau_0} \symup{e}^{-W/k_\text{B} T} \cdot \text{exp}\left( -\int_{T_0}^T (b\tau_0)^{-1} \symup{e}^{-W/k_\text{B}T'} \symup{d}T' \right).
\end{equation}
Für tiefe Temperaturen, also zum Beginn der Messung gilt $\symup{e}^{-W/k_\text{B}T} \approx 0$
\begin{equation*}
    \Rightarrow \int_{T_0}^T \symup{e}^{-W/k_\text{B}T'} \symup{d}T' \approx 0,
\end{equation*}
weshalb sich für den Anfangsbereich (für tiefe T)
\begin{equation}
    \label{eq:W1}
    \symup{ln}(i(T)) = const - \frac{W}{k_\text{B}T}
\end{equation}
schreiben lässt. Damit lässt sich durch eine lineare Regression des Anfangsbereichs die Aktivierungsenergie $W$ der Dipole bestimmen.
Des Weiteren ergibt sich ein von der Feldstärke $E$ unabhängiges Maximum des Stromverlaufes bei 
\begin{equation}
    \label{eq:T_max}
    T_\text{max}^2 = \frac{b W \tau(T_\text{max})}{k_\text{B}}
\end{equation}
durch Ableiten von \autoref{eq:i_T}. Mit dem Wertepaar $(T_\text{max}, \tau(T_\text{max}))$ lässt sich so ebenfalls die Aktivierungsenergie $W$ bestimmen.

\subsection{Herleitung über die Stromdichte}
\label{sec:Herleitung_Stromdichte}
Der zweiten Methode zur Herleitung des Depolarisationsstroms liegt die Überlegung zu Grunde, dass die zeitliche Änderungsrate der Polarisation gleich der Stromdichte $j$ ist:
\begin{equation}
    \label{eq:Ansatz2}
    \frac{\symup{d}P}{\symup{d}t} = \frac{P(T)}{\tau{T}} = j(T).
\end{equation}
Durch Integration folgt 
\begin{equation*}
    P(T) \cdot A = \int_T^{T_f} j(T') \cdot A \symup{d}T' = \int_{t(T)}^\infty i(t') \symup{d}t',
\end{equation*}
wobei $A$ die Fläche der Kontakte am Kristall ist. Zusammen mit \autoref{eq:Ansatz2} ergibt sich so 
\begin{align}
    \tau(T) &= \frac{\int_{t(T)}^{\infty} i(t') dt'}{i(T)} \nonumber \\
    \Leftrightarrow \symup{ln}(\tau(T)) &= \symup{ln}(\tau_0) + \frac{W}{k_\text{B}T} = \symup{ln}\left(\int_{i(T)}^\infty i(T')dT' \right) - \symup{ln}(i(T)).
    \label{eq:W_tau0}
\end{align}
Das Integral kann dabei über graphische Integration über $i(T)$ approximiert werden. Durch eine lineare Ausgleichsrechnung können $W$ und $\tau_0$ bestimmt werden.
