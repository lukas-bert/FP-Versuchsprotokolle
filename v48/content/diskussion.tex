\section{Diskussion}
\label{sec:Diskussion}
Ein Theoriewert für die benötigte Aktivierungsenergie $W$ des Kristalls liegt für strontiumdotierte Kaliumbromidkristalle bei rund $W=\qty{0.66}{\electronvolt}$~\cite{v48_litwert_W}. Die mit 
den zwei verschiedenen Methoden bestimmten Werte für die Aktivierungsenergie liegen allesamt im Bereich von etwa einem Elektronenvolt, explizit
\begin{align*}
    W_{\text{Anstieg}, 1} &= \qty{0.693+-0.022}{\electronvolt} \\
    W_{\text{Anstieg}, 2} &= \qty{0.744+-0.018}{\electronvolt} \\
    W_{\text{intfit}, 1} &= \qty{0.93+-0.06}{\electronvolt} \\    
    W_{\text{intfit}, 2} &= \qty{1.24+-0.10}{\electronvolt}. \\
\end{align*}
Es zeigt sich, dass die Größenordnung bei allen berechneten Werten übereinstimmt, wobei die Werte der Aktivierungsenergie $W$, welche über die Integralmethode 
berechnet werden, stärker abweichen.
Die Relaxationszeit $\tau$ ist in der Literatur mit $\tau=\qty{4e-14}{\second}$~\cite{v48_litwert_W} angeben. Experimentell bestimmt werden 
\begin{align*}
    \tau_{0, \text{Anstieg}_1} &= \qty{5.4(1.3)e-19}{\second} \\
    \tau_{0, \text{Anstieg}_2} &= \qty{5.3(1.0)e-19}{\second} \\
    \tau_{0, \text{intfit}_1} &= \qty{0.7(1.8)e-17}{\second} \\
    \tau_{0, \text{intfit}_2} &= \qty{1(5)e-23}{\second}. \\
\end{align*}
Hier zeigt sich, dass die Abweichungen mit bis zu 9 Größenordnungen als sehr groß zu bezeichnen sind.

Ein möglicher Grund für die großen Abweichungen ist die Tatsache, dass das benötigte Vakuum im Rezipienten erst kurz vor der Datennahme erzeugt wurde. Üblich 
ist es, dass eine durchgängig laufende Vakuumpumpe den Rezipienten permanent evakuiert und somit sicherstellt, dass sich eine möglichst geringe Menge Wasser
ansammelt. Da dies bei dieser Durchführung nicht der Fall war, ist eine Abweichung der Daten nicht auszuschließen.

Es ist ebenfalls wahrscheinlich, dass der Fit für den thermischen Hintergrund ungenügend ist. Ein Hinweis darauf ist der unphysikalische, negative Strom nach
Abzug des Hintergrundes. Diese Unsicherheit setzt sich in die Bestimmung der Aktivierungsenergie $W$ und der Relaxationszeit $\tau$ fort und potenziert sich dabei.

Ableseungenauigkeiten am Picoamperemeter sind gleichermaßen nicht auszuschließen und können sich aufgrund von Fehlerfortpflanzung auch zu großen Ungenauigkeiten
in der Berechnung der Endgrößen auswirken.

Insbesondere für die Integralmethode ist es essentiell, dass die Heizrate $b$ konstant ist. Unsicherheiten hier können auch zu Abweichungen im Endergebnis führen.