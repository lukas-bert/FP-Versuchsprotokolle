\section{Diskussion}
\label{sec:Diskussion}
Bei der Bestimmung der Landé-Faktoren in Abschnitt~\ref{sec:Landéfaktoren} wird nebenbei die Horizontalkomponente des Erdmagnetfeld als y-Achsenabschnitt der
Fits bestimmt. Für die beiden Isotope ergibt sich
\begin{align*}
    b_1 &= \qty{2.39(0.01)e-05}{\tesla} \\
    b_2 &= \qty{2.36(0.01)e-05}{\tesla} \\
\end{align*}
Die Werte liegen nah beieinander, sind jedoch nicht mit ihren jeweiligen Unsicherheiten miteinander vereinbar. Diese geringfügige Abweichung lässt sich etwa mit
Unsicherheiten bei der Vermessung der Transmissionspeaks erklären.

Aus den berechneten Landéfaktoren $g_F$ folgten die Kernspins
\begin{align*}
    I_1 &= \num{1.5339+-0.0012} \\
    I_2 &= \num{2.5515+-0.0018}.
  \end{align*}
Literaturwerte nach Referenz \cite{Rubidium} für die Kernspins von $\ce{^{85}Rb}$ und $\ce{^{87}Rb}$ sind $I=3/2$ und $I=5/2$. Auch hier wird eine geringfügige
Abweichung festgestellt, die mit nicht betrachteten Unsicherheiten beim Einstellen oder Ablesen der Magnetfeldstärken an den Potentiometern erklärt werden kann.
Eine Identifizierung der Isotope als $\ce{^{85}Rb}$ und $\ce{^{87}Rb}$ ist möglich.

Das Isotopenverhältnis in der untersuchten Probe wird in Abschnitt~\ref{sec:Isotoenverhältnis} bestimmt zu
\begin{equation*}
    \frac{\ce{^{85}Rb}}{\ce{^{87}Rb}} = \num{2}.
  \end{equation*}
Über die Periodendauer der Rabi-Oszillationen wird das Isotoenverhältnis ebenfalls bestimmt, wobei sich ein Wert von
\begin{equation*}
    \frac{\ce{^{85}Rb}}{\ce{^{87}Rb}} = \num{1.49+-0.08}.
  \end{equation*}
Diese Werte liegen soweit auseinander, dass sie nicht mit statistischen Unsicherheiten zu erklären sind. Denkbar ist ein Bedienfehler an der Messapparatur.
Ein Theoriewert für das natürlich vorkommenden Isotopenverhätnis ist als
\begin{equation*}
    \frac{\ce{^{85}Rb}}{\ce{^{87}Rb}} = \num{2.57}
  \end{equation*}
gegeben \cite{Rubidium}. Es ist jedoch zu beachten, dass in dem verwendeten Versuchsaufbau ein anderes Verhätnis der Rubidiumisotope verwendet wird.