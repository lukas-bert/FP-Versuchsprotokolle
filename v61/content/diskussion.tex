\section{Diskussion}
\label{sec:Diskussion}
Zuerst wurde die Stabilitätsbedingung des Lasers überprüft, welche sich für beide Spiegelkonfigurationen im messbaren Bereich bestätigt. \\
Die gemessene Intensitätsverteilung der TEM-Moden lässt sich durch die Theoriekurve dieser gut darstellen. Es konnten die $\text{TEM}_{00}$,
die $\text{TEM}_{10}$ und die $\text{TEM}_{20}$-Mode observiert werden. Die Intensitätsverteilung der $\text{TEM}_{10}$-Mode weist eine gewisse Asymmetrie vor,
die unter anderem durch Unregelmäßigkeiten des Tungsten Drahtes und weitere Einflüsse, wie Verschmutzungen auf der Streulinse, erklärt werden kann. \\
Durch die Polarisationsmessung ergibt sich eine annähernd lineare Polarisation des Lasers und die zu erwartende Periodizität. \\
Im Multimoden Betrieb des Lasers ergeben sich mit den eingestellten Resonatorlängen Frequenzabstände von $\qty{75}{\mega\hertz} < \symup{\Delta}f < \qty{310}{\mega\hertz}$.
Die zu erwartende Proportionalität nach \autoref{eq:delta_f} von $L$ zu $1/\symup{\Delta}f$ bestätigt sich experimentell, was auch anhand der experimentell bestimmten Resonatorlängen
deutlich wird, die mit den eingestellten Längen übereinstimmen.
Wird die Bandbreite des hier relevanten Neon-Übergangs von 
$\symup{\Delta}\nu = \qty{1.5}{\giga\hertz}$ \cite{HeNe_Wiki} betrachtet, so rechtfertigt dies den Multimoden-Betrieb des Lasers, da innerhalb der Bandbreite des Neon-Übergangs 
mehrere Moden mit Abstand $\symup{\Delta}f$ möglich sind. \\
Die Wellenlänge des Lasers konnte zu $\lambda_\text{exp} = \qty{634.5 +- 7.24}{\nano\metre}$ bestimmt werden. Der Literaturwert lautet 
$\lambda_\text{lit} =  \qty{632.8}{\nano\metre}$ \cite{HeNe_Wiki}, was innerhalb der Messunsicherheit des experimentell bestimmten Wertes liegt, wodurch 
die Wellenlänge mit guter Präzision ermittelt ist. \\
Zusammenfassend bestätigen sich die theoretischen Beschreibungen des Lasers durch die in diesem Versuch gemessenen Daten.
