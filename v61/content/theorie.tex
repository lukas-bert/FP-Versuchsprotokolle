\section{Zielsetzung}
In diesem Versuch wird die Funktionsweise eines Helium-Neon Lasers untersucht. Dafür werden nach einer Justage des Lasers 
verschiedene Eigenschaften des Laserlichts, wie die Wellenlänge oder TEM-Moden vermessen.

\section{Theorie}
\label{sec:Theorie}

Der Begriff \texttt{LASER} ist ein Akronym für \textit{Light amplification by stimulated emission of radiation}. Laser zeichnen als
leistungsstarke Quelle von monochromatischem Licht aus und finden daher häufig in verschiedenen Experimenten Anwendung. Die theoretischen
Grundlagen und der Aufbau eines Helium-Neon Lasers (HeNe-Laser) werden hier erörtert.

\subsection{Zustandssysteme und Besetzungsinversion}
Um die Funktionsweise eines Lasers zu vestehen, ist ein Blick auf die quantenmechanische Beschreibung von Zustandssystemen notwendig.
In der Quantenmechanik wird werden Zustände von Teilchen einer bestimmten Energie zugeordnet, sodass unterschiedliche Zustände verschiedene
Energien besitzen. Es ergibt sich ein Zustandssytem, in dem die Teilchen durch Absorbtion und Emission von Photonen in Zustände höherer Energie wechseln können,
wenn die Energie der anregenden (Quasi-)Teilchen
\begin{equation}
    E = h\nu
\end{equation}
genau der Energiedifferenz zweier Zustände entspricht.

Befinden sich mehr Teilchen in einem höheren Energiezustand als dem Grundzustand, wird von \textit{Besetzungsinversion} gesprochen. Diese kann
nur für ein Zustandssystem mit mehr als 2 Zuständen erreicht werden, da hier die Übergangswahrscheinlichkeiten $E_1 \rightarrow E_2 ≤ E_2 \rightarrow E_1$
maximal eine Gleichverteilung zulassen.

In einem Zustandssystem mit mehr als zwei Energieniveaus ist es hingegen möglich, durch hinreichende äußere Anregung eine Besetzungsinversion zu erreichen. 
Die äußere Anregung wird auch als \textit{pumpen} bezeichnet.

\subsection{Zustandssystem von Helium und Neon}
In \autoref{fig:Zustandssystem_HeNe} ist das Zustandssystem von Helium und Neon gezeigt. 

\subsection{Spontane und stimulierte Emission}

\subsection{Resonator}

\begin{equation}
    g_{\symup{i}} = \frac{L}{r_{\symup{i}}}
    \label{eq:g_i}
\end{equation}

\begin{equation}
    0 ≤ g_1g_2 ≤ 1
    \label{eq:g1g2}
\end{equation}

\subsection{TEM-Moden}

\subsection{Brechung am Gitter}

\begin{equation}
    n \lambda = g \sin(\varphi_n)
    \label{eq:Interferenzbedingung}
\end{equation}

\begin{equation}
    \lambda = \frac{g\sin\left(\tan\left(\frac{s_{\symup{n}}}{2d}\right)\right)}{n}
    \label{eq:lambda}
\end{equation}