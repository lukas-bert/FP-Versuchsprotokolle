\section{Zielsetzung}
\label{sec:Zielsetzung}
Die effektive Masse beschreibt in der Festkörperphysik die scheinbare Masse von Teilchen in einem Kristall. In diesem Versuch wird die effektive Masse der Leitungselektronen von
n-dotiertem Galliumarsenid (n-GaAs) mithife des Effekts der Faradyrotation bestimmt. Der Faraday-Effekt bezeichnet die Drehung der Polarisationsebene von linear polarisiertem Licht
beim Durchgang durch ein Medium in einem Magnetfeld.

\section{Theorie}
\label{sec:Theorie}
\subsection{Bandstruktur}
\label{subsec:Bandstruktur}

\subsection{Effektive Masse}
\label{subsec:Effektive Masse}

\subsection{Zirkulare Doppelbrechung}
\label{subsec:Zirkulare Doppelbrechung}

\subsection{Faraday-Effekt}
\label{subsec:Faraday-Effekt}

