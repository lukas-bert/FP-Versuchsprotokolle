\section{Diskussion}
\label{sec:Diskussion}
Zuerst wurde die maximale magnetische Kraftflussdichte im Probenbereich ermittelt. 
An \autoref{fig:magnetfeld} lässt sich erkennen, dass die Messwerte eine symmetrische Kurve ergeben, was auf eine qualitative Messung hindeutet.
Bei der Bestimmung der effektiven Elektronenmasse von Galliumarsenid ergeben sich die Werte  $m_1 = \qty{6.69+-0.82e-32}{\kilo\gram}$ und $m_2 =  \qty{8.21+-0.96e-32}{\kilo\gram}$.
Der Literaturwert beträgt $m^* = \num{0.063}m_e \approx \qty{5.74e-32}{\kilo\gram}$, woraus relative Abweichungen von $\qty{16+-14}{\percent}$ für den ersten Messwert und 
$\qty{43+-16}{\percent}$ für den zweiten Messwert folgen. Die relative Abweichung des Mittelwertes lautet $\symup{\Delta}_\text{rel} (\overline{m}^*)= \qty{29+-11}{\percent}$.
Eine Ursache für die Abweichungen könnte in der Genauigkeit der Winkelbestimmung 
liegen. Anhand des Graphen des verwendeten Oszilloskops ist es nur schwer möglich, eine exakte Position eines Minimums festzustellen, weshalb die Messdaten der Winkel mit einer 
gewissen Unsicherheit belastet sind. Auch die Tatsache, dass die Ausgleichsgeraden zur Bestimmung des Proportionalitätsfaktors einen von Null verschiedenen $y$-Achsenabschnitt 
aufweisen, deutet auf systematische Abweichungen hin.
Im Rahmen der Messgenauigkeit bestätigen die experimentell ermittelten Werte jedoch den Theoriewert.
Die effektive Masse der Leitungselektronen in Galliumarsenid ist erfolgreich bestimmt.
