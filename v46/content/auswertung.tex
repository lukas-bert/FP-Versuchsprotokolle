\section{Auswertung}
\label{sec:Auswertung}

\subsection{Bestimmung der maximalen magnetischen Kraftflussdichte}
Zur Bestimmung der maximalen magnetischen Flussdichte werden die in \autoref{fig:magnetfeld} dargestellten Messwerte verwendet. Die genauen Messdaten sind im Anhang
\ref{subsec:Originaldaten} aufgelistet. Anhand des Graphen lässt sich ein eindeutiges Maximum erkennen. Dieses liegt bei einer magnetischen Flussdichte von 
$B = \qty{412}{\milli\tesla}$.

\begin{figure}
  \centering
  \includegraphics[width=.8\textwidth]{magnetfeld.pdf}
  \caption{Grafische Darstellung der Messdaten zur Bestimmung der magnetischen Kraftflussdichte.}
  \label{fig:magnetfeld}
\end{figure}

\subsection{Bestimmung der effektiven Masse der Leitungselektronen}
\label{subsec:Elektronenmasse}
Zur Bestimmung der effektiven Elektronenmasse werden neun Interferenzfilter und drei verschiedene Galliumarsenid Proben verwendet. Eine der Proben ist undotiert und wird verwendet 
um die Faradayrotation gebundender Ladungsträger zu bestimmen. Die anderen Proben haben Donatorenkonzentrationenn von $\qty{1.2e18}{\centi\metre^{-3}}$ (Probe $1$) und 
$\qty{2.8e18}{\centi\metre^{-3}}$ (Probe $2$). Die undotierte Kontrollprobe heiße Probe $0$.
Die Dicken der Proben sind als
\begin{align*}
  d_0 &= \qty{5.11}{\milli\metre} & d_1 &= \qty{1.36}{\milli\metre} & d_2 &= \qty{1.296}{\milli\metre}
\end{align*}
gegeben. Die jeweiligen Messwerte der Drehwinkel $\theta_1$ und $\theta_2$, sowie die daraus berechnete Differenz $\theta$ sind den Tabellen 
\ref{tab:mw1} - \ref{tab:mw3} zu entnehmen.

\begin{table}
  \centering
  \caption{Messwerte zur undotierten Probe und nach \autoref{eqn:theta_diff} bestimmter Drehwinkel der Faradayrotation.}
  \label{tab:mw1}
  \begin{tabular}{c c c c}
    \toprule
    $\lambda \mathbin{/} \unit{\micro\meter}$ & $\theta_{1} \mathbin{/} \unit{\degree}$ & $\theta_{2} \mathbin{/} \unit{\degree}$ &%
     $\theta \mathbin{/} \unit{\degree}$ \\
    \midrule
    $1,06 $ & $94,28$ & $72,58$ & $10,85$ \\
    $1,29 $ & $91,35$ & $75,07$ & $ 8,14$ \\
    $1,45 $ & $90,50$ & $80,73$ & $ 4,88$ \\
    $1,72 $ & $86,05$ & $80,00$ & $ 3,02$ \\
    $1,96 $ & $80,73$ & $75,53$ & $ 2,60$ \\
    $2,156$ & $80,28$ & $74,38$ & $ 2,95$ \\
    $2,34 $ & $55,00$ & $48,00$ & $ 3,50$ \\
    $2,51 $ & $22,52$ & $30,20$ & $-3,84$ \\
    $2,65 $ & $71,48$ & $65,17$ & $ 3,16$ \\
    \bottomrule
  \end{tabular}
\end{table}

\begin{table}
  \centering
  \caption{Messwerte zur Probe mit $N = \qty{1.2e18}{\centi\metre^{-3}}$ und nach \autoref{eqn:theta_diff} bestimmter Drehwinkel der Faradayrotation.}
  \label{tab:mw2}
  \begin{tabular}{c c c c}
    \toprule
    $\lambda \mathbin{/} \unit{\micro\meter}$ & $\theta_{1} \mathbin{/} \unit{\degree}$ & $\theta_{2} \mathbin{/} \unit{\degree}$ &%
     $\theta \mathbin{/} \unit{\degree}$ \\
    \midrule
    $1,06 $ & $87,17$ & $80,00$ & $3,58$ \\
    $1,29 $ & $87,00$ & $80,17$ & $3,42$ \\
    $1,45 $ & $89,00$ & $84,13$ & $2,43$ \\
    $1,72 $ & $87,20$ & $80,13$ & $3,53$ \\
    $1,96 $ & $82,03$ & $78,02$ & $2,01$ \\
    $2,156$ & $81,15$ & $75,18$ & $2,98$ \\
    $2,34 $ & $54,70$ & $46,57$ & $4,07$ \\
    $2,51 $ & $35,30$ & $27,22$ & $4,04$ \\
    $2,65 $ & $73,52$ & $65,07$ & $4,23$ \\
    \bottomrule
  \end{tabular}
\end{table}

\begin{table}
  \centering
  \caption{Messwerte zur Probe mit $N = \qty{2.8e18}{\centi\metre^{-3}}$ und nach \autoref{eqn:theta_diff} bestimmter Drehwinkel der Faradayrotation.}
  \label{tab:mw3}
  \begin{tabular}{c c c c}
    \toprule
    $\lambda \mathbin{/} \unit{\micro\meter}$ & $\theta_{1} \mathbin{/} \unit{\degree}$ & $\theta_{2} \mathbin{/} \unit{\degree}$ &%
     $\theta \mathbin{/} \unit{\degree}$ \\
    \midrule
    $1,06 $ & $91,00$ & $78,62$ & $6,19$ \\
    $1,29 $ & $88,83$ & $78,25$ & $5,29$ \\
    $1,45 $ & $90,07$ & $81,90$ & $4,08$ \\
    $1,72 $ & $85,02$ & $79,68$ & $2,67$ \\
    $1,96 $ & $82,13$ & $74,15$ & $3,99$ \\
    $2,156$ & $83,58$ & $70,57$ & $6,51$ \\
    $2,34 $ & $56,48$ & $43,52$ & $6,48$ \\
    $2,51 $ & $39,10$ & $25,08$ & $7,01$ \\
    $2,65 $ & $79,10$ & $64,03$ & $7,53$ \\
    \bottomrule
  \end{tabular}
\end{table}

Die Messwerte der Drehwinkel $\theta$ werden normiert, indem sie durch die jeweiligen Probenlängen geteilt werden. Durch Subtraktion der Messwerte der undotierten Probe von 
denjenigen der n-dotierten Proben ergibt sich die Faradayrotation pro Eineheitslänge 
\begin{equation*}
  \theta_{\text{frei}, i} = \frac{\theta_i}{d_i} - \frac{\theta_0}{d_0}, \qquad i = \{1, 2\}.
\end{equation*}
An \autoref{eqn:theta_l2} lässt sich erkennen, dass $\theta_\text{frei}$ direkt proportional zum Quadrat der Wellenlänge ($\lambda^2$) ist. Mithilfe des Proportionalitätsfaktors 
kann die effektive Elektronenmasse~$m^*$ ermittelt werden. Dazu wird eine lineare Funktion der Form $f(x) = ax + b$ mit $f(x) = \theta_\text{frei}$ und $x = \lambda^2$
angesetzt. In der Theorie gilt $b = 0$, jedoch wird hier der zusätzliche Freiheitsgrad $b$ dennoch verwendet, um eventuelle systematische Abweichungen zu kompensieren.
Es wird eine lineare Regression mittels \textit{scipy} \cite{scipy} durchgeführt. Die Messwerte der Faradayrotation pro Eineheitslänge und die Ausgleichsgeraden sind für 
beide Galliumarsenid-Proben in \autoref{fig:fit} dargestellt.

\begin{figure}
  \centering
  \includegraphics{fit.pdf}
  \caption{Werte der Faradayrotation pro Eineheitslänge $\theta_\text{frei}$ der dotierten Proben und ermittelte Ausgleichsgeraden.}
  \label{fig:fit}
\end{figure}

Die Parameter der Geradengleichung ergeben sich zu
\begin{align*}
  a_1 &= \qty{7.186+-1.752e12}{\metre^{-3}} &  a_2 &= \qty{11.14+-2.61e12}{\metre^{-3}} \\
  b_1 &= \qty{1.76+-7.67}{\metre^{-1}} & b_1 &= \qty{17.56+-11.41}{\metre^{-1}}
\end{align*}
für Probe $1$ und Probe $2$ respektive. Durch Umstellen von \autoref{eqn:theta_l2} und Einsetzen des Ansatzes $\theta_\text{frei} = a \lambda^2$ ergibt sich 
die Gleichung
\begin{equation}
  m^* = \sqrt{\frac{e^3 N B}{8 \symup{\pi}^2 \varepsilon_0 c^3 n a}}
\end{equation}
zur Berechnung der effektiven Elektronenmasse. Für die magnetische Kraftflussdichte wird $B = \qty{412}{\milli\tesla}$ verwendet. Der Berechnungsindex von Galliumarsenid ist
im vorliegenden Wellenlängenbereich $n = \num{3.354}$ \cite{Brechungsindex_GaAs}. Für die bekannten Werte der Elementarladung $e$, der Permittivität des Vakuums $\varepsilon_0$ und der 
Lichtgeschwindigkeit $c$ werden Werte aus \textit{scipy constants} \cite{scipy} verwendet. Für die einzelnen Proben ergeben sich nach Gaußscher Fehlerfortpflanzung 
die Werte
\begin{align*}
  m_1 &= \qty{6.69+-0.82e-32}{\kilo\gram} & m_2 &=  \qty{8.21+-0.96e-32}{\kilo\gram}
\end{align*}
für die effektive Elektronenmasse. Der Mittelwert der Messungen ergibt sich zu $\qty{7.45+-0.63e-32}{\kilo\gram}$.
