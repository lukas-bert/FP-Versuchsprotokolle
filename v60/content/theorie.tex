\section{Zielsetzung}
Laser (\textbf{l}ight \textbf{a}mplification by \textbf{s}timulated \textbf{e}mission of \textbf{r}adiation) sind in der experimentellen Physik von zentraler Bedeutung
und werden u.a. zur Untersuchung atomarer Strukturen genutzt. Sie bieten eine leistungsstarke Quelle für kohärentes Licht und sind 
oft auf bestimmte Wellenlängen einstellbar. In diesem Versuch wird die Funktionsweise eines Diodenlasers, die Justierung des Lasers
und dessen Bedienung im Experiment am Beispiel des Absorptionsspektrums von Rubidium erprobt.


\section{Theorie}
\label{sec:Theorie}

\cite{sample}
