\section{Zielsetzung}
\label{sec:Zielsetzung}
Kosmische Myonen entehen in der Erdatmosphäre auf einer Höhe von rund $\qty{10}{\kilo\metre}$ durch den Zerfall von Pionen. Ziel des Versuchs ist es, die 
mittlere Lebensdauer der Myonen mithilfe eines geeigneten Versuchsaufbaus zu ermitteln.


\section{Theorie}
\label{sec:Theorie}
Trifft ein hochenergetisches Proton aus dem Weltraum auf ein Luftmolekül der Erdatmosphäre kann ein Pion entstehen. Pionen haben eine kurze mittlere 
Lebensdauer von $\tau_\pi = \qty{26}{\nano\second}$ \cite{PDG:muon} und zerfallen hauptsächlich in Myonen gemäß

\begin{align*}
    \pi^+ &\to \mu^+ + \nu_\mu \\
    \pi^- &\to \mu^- + \bar{\nu}_\mu.
\end{align*}

Die Myonen bewegen sich mit annähernd Lichtgeschwindigkeit und zerfallen überwiegend über 
\begin{align*}
    \mu^+ &\to e^+ + \nu_e + \bar{\nu}_\mu \\
    \mu^- &\to e^- + \bar{\nu}_e + \nu_\mu
\end{align*}
zu Elektronen und den entsprechenden Neutrinos.

\subsection{Detektion der Myonen}
\label{subsec:Detektion der Myonen}
In diesem Versuch wird die Lebensdauer einzelner Myonen mithilfe eines Szintillationsdetektors ermittelt. Tritt ein hochenergetisches Myon in das Szintillatormaterial
ein, wird ein Großteil der Energie des Myons dort deponiert. Dies hat zur Folge, dass die Szintillatormoleküle angeregt werden. Nach einer kurzen Relaxationszeit
wird ein Photon emittiert, welches mithilfe eines Photomultipliers detektiert werden kann. Der Detektor ist in Abschnitt \ref{subsec:Aufbau} im Detail beschrieben.

Zerfällt ein Myon in dem Szintillationsdetektor werden die Szintillatormoleküle erneut angeregt und ein Lichtblitz kann detektiert werden. Durch Messung der Zeitpunkte des Eintritts
und Zerfalls eines Myons lässt sich auf die Lebensdauer rückschließen. Geschieht dies für viele Myonen lässt sich die mittlere Lebensdauer bestimmen.

\subsection{Lebensdauer der Myonen}
\label{subsec:Lebensdauer der Myonen}
In einem infinitesimalen Zeitintervall $\symup{d}t$ zerfällt ein Myon mit derselben Wahrscheinlichkeit. Dadurch ergibt sich der Zusammenhang
\begin{equation*}
    \symup{d}W = \lambda \symup{d}t
\end{equation*}
wobei $\symup{d}W$ die Wahrscheinlichkeit bezeichtet, dass das Myon in einem infinitesimalen Bereich zerfällt und der Proportionalitätsfaktor $\lambda$ als Zerfallskonstante
indentifiziert wird.
Werden $N$ Teilchen betrachet, ergibt sich die Wahrscheinlichkeit, dass die Teilchen noch nicht zerfallen sind von
\begin{equation*}
    \symup{d}N = -N \symup{d}W = -N \lambda \symup{d}t.
\end{equation*} 
Durch Umstellen und Lösen der Differentialgleichung lässt sich das Zerfallsgesetz
\begin{equation*}
    N(t)=N_0\symup{e}^{-\lambda t}
\end{equation*}
herleiten. Hierbei wird die Anzahl der Teilchen, die zu Beginn betrachtet werden, als $N_0$ bezeichnet. Über die zeitliche Ableitung und Umstellen lässt sich eine Verteilung für
die Lebensdauern berechnen
\begin{equation*}
    \frac{N(t)}{N_0}=\lambda\symup{e}^{-\lambda t} \symup{d}t.
\end{equation*}
Der Erwartungswert bezüglich der Zeit dieser Exponentialverteilung ist als mittlere Lebensdauer zu interpretieren
\begin{equation}
    <t> = \tau = \int_0^\infty \lambda\symup{e}^{-\lambda t} \symup{d}t = \frac{1}{\lambda}.
    \label{eq:Lebensdauer}
\end{equation}
