\section{Zielsetzung}
\label{sec:Zielsetzung}
Kosmische Myonen entehen in der Erdatmosphäre auf einer Höhe von rund $\qty{10}{\kilo\metre}$ durch den Zerfall von Pionen. Ziel des Versuchs ist es, die 
mittlere Lebensdauer der Myonen mithilfe eines geeigneten Versuchsaufbaus zu ermitteln.


\section{Theorie}
\label{sec:Theorie}
Trifft ein hochenergetisches Proton aus dem Weltraum auf ein Luftmolekül der Erdatmosphäre kann ein Pion entstehen. Pionen haben eine kurze mittlere 
Lebensdauer von $\tau_\pi = \qty{26}{\nano\second}$ und zerfallen hauptsächlich in Myonen gemäß

\begin{align*}
    \pi^+ &\to \mu^+ + \nu_\mu \\
    \pi^- &\to \mu^- + \bar{\nu}_\mu.
\end{align*}

Die Myonen bewegen sich mit annähernd Lichtgeschwindigkeit und zerfallen überwiegend über 
\begin{align*}
    \mu^+ &\to e^+ + \nu_e + \bar{\nu}_\mu \\
    \mu^- &\to e^- + \bar{\nu}_e + \nu_\mu
\end{align*}
zu Elektronen und den entsprechenden Neutrinos.

\cite{Wermes}
\cite{V01}
