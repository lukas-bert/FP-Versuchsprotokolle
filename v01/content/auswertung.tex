\section{Auswertung}
\label{sec:Auswertung}
Die Auswertung dieses Versuchs erfolgt mit zur Verfügung gestellten Ersatzdaten, da die eigentliche Messung auf Grund eines technischen Problemes 
nicht durchgeführt werden konnte. Lediglich die Messdaten zur Bestimmung der Halbwertsbreite der Messkurve der Verzögerungszeiten wurde im Rahmen der Versuchsdurchführung aufgenommen.

\subsection{Bestimmung der Halbwertsbreite der Messkurve der Verzögerungszeiten}
Zur Bestimmung der Halbwertsbreite werden die nach der Koinzidenz gemessenen Zählraten der beiden Photomultiplier betrachtet, welche in \autoref{tab:Mess1} gelistet sind. 
\begin{table}
  \tiny
  \centering
  \caption{Zählraten $N$ der Koinzidenz gegen verschiedene Verzögerungszeiten $T_\text{VZ}$ der Leitungen. Eine negative Verzögerungszeit beschreibt eine größere Verzögerung der Leitung zum linken PMT.
  Es sind zwei Messreihen zu verschiedenen Zählzeiten aufgetragen.}
  \label{tab:Mess1}
  \begin{tabular}{c c c}
    \toprule
    {$ T_\text{VZ} \mathbin{/} \unit{\nano\second}$} & {$N \mathbin{/} \qty{10}{\second}$} & {$N \mathbin{/} \qty{20}{\second}$} \\
    \midrule
     0 & 59 & 42 \\
    -1 & 76 &  \\
    -2 & 42 & 28 \\
    -3 & 73 &  \\
    -4 & 64 & 36 \\
    -5 & 58 &  \\
    -6 & 49 & 19 \\
    -7 & 63 &  \\
    -8 & 40 & 23 \\
    -9 & 40 &  \\
   -10 & 23 & 9 \\
   -11 & 18 &  \\
   -12 & 11 & 4 \\
   -13 &  6 &  \\
   -14 &  0 & 0 \\
     1 & 65 &  \\
     2 & 48 & 21 \\
     3 & 45 &  \\
     4 & 51 & 16 \\
     5 & 30 &  \\
     6 & 32 & 18 \\
     7 & 33 &  \\
     8 & 18 & 9 \\
     9 & 11 &  \\
    10 &  7 & 3 \\
    11 &  4 &  \\
    12 &  4 & 0 \\
    13 &  0 &  \\
    \bottomrule
  \end{tabular}
\end{table}
In \autoref{fig:T_VZ} sind 
die Zählraten in Abhängigkeit zur Verzögerungszeit $T_\text{VZ}$ aufgetragen. Als Zählraten-Experiment genügen die Messwerte einer Poissonverteilung. Der statistische 
Fehler lautet somit $\sqrt{N}$. 
Eine negative Verzögerungszeit bedeutet hier eine größere Verzögerung der 
Leitung zum linken PMT, während eine positive Verzögerungszeit die Verzögerungsleitung des rechten PMT beschreibt. Die Diskriminatoren sind auf eine Pulsbreite 
von $\qty{10}{\nano\second}$ eingestellt. 
\begin{figure}[H]
  \centering
  \includegraphics[width = .8\textwidth]{plot1.pdf}
  \caption{Zählraten nach der Koinzidenz gegen verschiedene Zeiten der Verzögerungsleitungen. Es sind die Messwerte aus zwei 
  verschiedenen Messreihen aufgetragen (rot: $\qty{20}{\second}$ Messintervall, schwarz: $\qty{10}{\second}$ Messintervall). Das Plateu und die Halbwertsbreite der Kurve sind markiert.}
  \label{fig:T_VZ}
\end{figure}
Um die Halbwertsbreite der Verteilung zu bestimmen wird der Bereich zwischen $\qty{-5}{\nano\second}$ und $\qty{1}{\nano\second}$ als Plateubereich betrachtet und 
der Mittelwert der Zählraten dieser Messdaten gebildet. Durch das Einzeichnen einer Horizontalen auf halber Höhe des Mittelwerts wird graphisch die Halbwertsbreite 
abgelesen. Es ergibt sich ein Wert von $\symup{\Delta}t = \qty{14}{\nano\second}$.

\subsection{Kalibrierung des Vielkanalanalysators}
Um die Kanalnummern des Vielkanalanalysators (VKA) einer Lebensdauer eines gemessenen Myons zuordnen zu können, wird eine Kalibrierungsmessung des VKA durchgeführt.
Die angesprochenen Kanalnummern sind in \autoref{tab:Mess2} gegen die eingestellten Impulsabstände gelistet und in \autoref{fig:Kalibrierung} graphisch aufgetragen. 
\begin{table}
  \centering
  \caption{Angesprochene Kanalnummern gegen eingestellte Impulsabstände.}
  \label{tab:Mess2}
  \begin{tabular}{S[table-format = 3.0] S[table-format = 1.1]}
    \toprule
    {Kanal} & {$t \mathbin{/} \unit{\micro\second}$} \\
    \midrule
     37 & 0,8 \\
     81 & 1,8 \\
    126 & 2,8 \\
    171 & 3,8 \\
    216 & 4,8 \\
    261 & 5,8 \\
    306 & 6,8 \\
    350 & 7,8 \\
    395 & 8,8 \\
    440 & 9,8 \\
    \bottomrule
  \end{tabular}
\end{table}
Mit Hilfe einer linearen Ausgleichsrechnung kann so eine Umrechnungsfunktion der Kanalnummern in Zeiten ermittelt werden. 
\begin{figure}[H]
  \centering
  \includegraphics[width = .7\textwidth]{plot2.pdf}
  \caption{Lineare Ausgleichsrechnung zur Kalibrierung der Kanalnummern des VKA. 
  Es sind eingestellte Impulsabstände gegen die Kanalnummer aufgetragen. Die lineare Regression wurde mit \textit{scipy} \cite{scipy} durchgeführt.}
  \label{fig:Kalibrierung}
\end{figure}
Mittels \textit{scipy} \cite{scipy} werden die Geradenparameter bestimmt. Es ergibt sich die Funktion
\begin{equation*}
  t [\unit{\micro\second}] = \num{0.0223 +- 0.00002} \cdot K - \num{0.017 \pm 0.003}
\end{equation*}
zur Umrechnung der Kanalnummer $K$ in einen zeitlichen Messwert.

\subsection{Bestimmung der Lebensdauer kosmischer Myonen}
Die Messdaten zur Bestimmung der Lebensdauer der Myonen liegen in Zählraten pro Kanal vor. Anhand der soeben ermittelten Umrechnungsfunktion werden die Kanalnummern 
in Zeiten konvertiert. Die daraus folgenden Datenpunkte sind in \autoref{fig:fit} und in \autoref{fig:fit_log} logarithmisch dargestellt.
Die Zählraten der Myonen folgen einem Exponentialgesetz.
Es wird der Ansatz
\begin{equation}
  \label{eqn:Ansatz}
  N(t) = N_0 \mathrm{e}^{-t/\tau} + U_0
\end{equation}
gewählt. Dabei wird mit dem Parameter $U_0$ eine konstante Untergrundrate berücksichtigt.
\begin{figure}
  \centering
  \includegraphics[width = .8\textwidth]{fit.pdf}
  \caption{Messdaten zur Bestimmung der mittleren Lebensdauer von Myonen. Die gemessenen Zählraten $N$ sind gegen die Lebensdauer aufgetragen. 
  Die Ausgleichsrechnung wird mit \textit{scipy} \cite{scipy} durchgeführt.}
  \label{fig:fit}
\end{figure}
Durch eine nicht-lineare Ausgleichsrechnung mittels \textit{scipy} \cite{scipy} ergeben sich die Werte
\begin{align*}
  N_0 &= \num{287 +- 2}, & \tau &= \qty{2.04 +- 0.03}{\micro\second}, & U_0 &= \num{3.65 +- 0.77}
\end{align*}
für die freien Parameter. Die Exponetialfunktion \ref{eqn:Ansatz} ist ebenfalls in den Abbildungen \ref{fig:fit} und \ref{fig:fit_log} eingezeichnet.
\begin{figure}
  \centering
  \includegraphics[width = .8\textwidth]{fit_log.pdf}
  \caption{Messdaten zur Bestimmung der mittleren Lebensdauer von Myonen in halblogarithmischer Darstellung. Die gemessenen Zählraten $N$ sind gegen die Lebensdauer aufgetragen. 
  Die Ausgleichsrechnung wird mit \textit{scipy} \cite{scipy} durchgeführt.}
  \label{fig:fit_log}
\end{figure}
