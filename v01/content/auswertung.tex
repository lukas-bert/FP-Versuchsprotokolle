\section{Auswertung}
\label{sec:Auswertung}
Die Auswertung dieses Versuchs erfolgt mit zur Verfügung gestellten Ersatzdaten, da die eigentliche Messung auf Grund eines technischen Problemes 
nicht durchgeführt werden konnte. Lediglich die Messdaten zur Bestimmung der Auflösungszeit wurden im Rahmen der Versuchsdruchführung aufgenommen.

\subsection{Bestimmung der Auflösungszeit}
Zur Bestimmung der Auflösungszeit, werden die nach der Koinzidenz gemessenen Zählraten der beiden Photomultiplier betrachtet. In \autoref{fig:T_VZ} sind 
die Zählraten in Abhängigkeit zur Verzögerungszeit $T_\text{VZ}$ aufgetragen. Als Zählraten-Experiment genügen die Messwerte einer Poissonverteilung. Der statistische 
Fehler lautet somit $\sqrt{N}$. 
Eine negative Verzögerungszeit bedeutet hier eine größere Verzögerung der 
Leitung zum linken PMT, während eine positive Verzögerungszeit die Verzögerungsleitung des rechten PMT beschreibt. Die Diskriminatoren sind auf eine Pulsbreite 
von $\qty{10}{\nano\second}$ eingestellt. 
\begin{figure}
  \centering
  \includegraphics[width = .8\textwidth]{plot1.pdf}
  \caption{Messdaten zur Bestimmung der Auflösungszeit der Koinzidenz. Es sind die Messwerte aus zwei 
  verschiedenen Messreihen aufgetragen und das Plateu, sowie die Halbwertsbreite der Kurve markiert.}
  \label{fig:T_VZ}
\end{figure}
Nach Bildung des Mittelwertes der Messwerte im Plateubereich ergibt sich eine Halbwertsbreite von etwa $\qty{14}{\nano\second}$.
Die Auflösungszeit berechnet sich damit zu 
\begin{equation*}
  \Delta t_\text{K} = 2 \cdot \qty{10}{\nano\second} - \qty{14}{\nano\second} = \qty{6}{\nano\second}.
\end{equation*}

\subsection{Kalibrierung des Vielkanalanalysators}
Um die Kanalnummern des Vielkanalanalysators einer Lebensdauer eines gemessenen Myons zuordnen zu können, wird eine Kalibrierungsmessung des VKA durchgeführt.
Die angesprochenen Kanalnummern sind in \autoref{fig:Kalibrierung} gegen die eingestellten Impulsabstände aufgetragen. Mit Hilfe einer linearen Ausgleichsrechnung
kann so eine Umrechnungsfunktion der Kanalnummern in Zeiten ermittelt werden. 
\begin{figure}
  \centering
  \includegraphics[width = .8\textwidth]{plot2.pdf}
  \caption{Lineare Ausgleichsrechnung zur Kalibrierung der Kanalnummern des VKA. 
  Es sind eingestellte Impulsabstände gegen die Kanalnummer aufgetragen. Die lineare Regression wurde mit \textit{scipy} \cite{scipy} durchgeführt.}
  \label{fig:Kalibrierung}
\end{figure}
Mittels \textit{scipy} \cite{scipy} werden die Geradenparameter bestimmt. Es ergibt sich die Funktion
\begin{equation*}
  t [\unit{\micro\second}] = \num{0.0223 +- 0.00002} \cdot K - \num{0.017 \pm 0.003}
\end{equation*}
zur Umrechnung der Kanalnummer $K$ in einen zeitlichen Messwert.

\subsection{Bestimmung der Lebensdauer kosmischer Myonen}
Die Messdaten zur Bestimmung der Lebensdauer der Myonen liegen in Zählraten pro Kanal vor. Anhand der soeben ermittelten Umrechnungsfunktion werden die Kanalnummern 
in Zeiten konvertiert. Die daraus folgenden Datenpunkte sind in \autoref{fig:fit} und in \autoref{fig:fit_log} logarithmisch dargestellt.
Die Zählraten der Myonen folgen einem Exponentialgesetz.
Es wird der Ansatz
\begin{equation}
  \label{eqn:Ansatz}
  N(t) = N_0 \mathrm{e}^{-t/\tau} + U_0
\end{equation}
gewählt. Dabei wird mit dem Parameter $U_0$ eine konstante Untergrundrate berücksichtigt.
\begin{figure}
  \centering
  \includegraphics[width = .8\textwidth]{fit.pdf}
  \caption{Messdaten zur Bestimmung der mittleren Lebensdauer von Myonen. Die gemessenen Zählraten $N$ sind gegen die Lebensdauer aufgetragen. 
  Die Ausgleichsrechnung wird mit \textit{scipy} \cite{scipy} durchgeführt.}
  \label{fig:fit}
\end{figure}
Durch eine nicht-lineare Ausgleichsrechnung mittels \textit{scipy} \cite{scipy} ergeben sich die Werte
\begin{align*}
  N_0 &= \num{287 +- 2}, & \tau &= \qty{2.04 +- 0.03}{\micro\second}, & U_0 &= \num{3.65 +- 0.77}
\end{align*}
für die freien Parameter. Die Exponetialfunktion \ref{eqn:Ansatz} ist ebenfalls in den Abbildungen \ref{fig:fit} und \ref{fig:fit_log} eingezeichnet.
\begin{figure}
  \centering
  \includegraphics[width = .8\textwidth]{fit_log.pdf}
  \caption{Messdaten zur Bestimmung der mittleren Lebensdauer von Myonen in halblogarithmischer Darstellung. Die gemessenen Zählraten $N$ sind gegen die Lebensdauer aufgetragen. 
  Die Ausgleichsrechnung wird mit \textit{scipy} \cite{scipy} durchgeführt.}
  \label{fig:fit_log}
\end{figure}
