\section{Diskussion}
\label{sec:Diskussion}
Der in diesem Experiment ermittelte Wert der Lebensdauer kosmischer Myonen lautet $\tau_\text{exp} = \qty{2.04 +- 0.03}{\micro\second}$. Der Literaturwert ist 
durch $\tau_\text{Lit} = \qty{2.1969811+-0.0000022}{\micro\second}$~\cite{PDG:muon} gegeben. Dies bedeutet eine Abweichung von $\qty{7.1}{\percent}$, die sich durch 
\begin{equation}
\symup{\Delta_\text{rel.}}(\tau) = \frac{|\tau_\text{Lit} - \tau_\text{exp}|}{\tau_\text{Lit}}
\end{equation}
berechnen lässt. Diese Abweichung lässt sich nicht mit der angegebenen -~rein statistischen~- Unsicherheit des Messwertes erklären. Eine potentielle Fehlerquelle
ist die Bildung myonischer Atome, die eine kürzere Lebensdauer als die kosmischen Myonen aufweisen, was die Abweichung zu kleineren Werten erklären könnte.
Ebenfalls werden statistische Unsicherheiten aus der Umrechnung der Kanalnummern in Messzeiten nicht berücksichtigt. 
Bei der Bestimmung der Kalibrierung der 
Koinzidenz konnte eine Auflösungszeit von $\qty{6}{\nano\second}$ festgestellt werden. Diese kann jedoch aus den in \autoref{sec:Auswertung} genannten Gründen
nicht mit den anderen Messergebnissen in Verbindung gebracht werden.
