\section{Diskussion}
\label{sec:Diskussion}
Der in diesem Experiment ermittelte Wert der Lebensdauer kosmischer Myonen lautet $\tau_\text{exp} = \qty{2.04 +- 0.03}{\micro\second}$. Der Literaturwert ist 
durch $\tau_\text{Lit} = \qty{2.1969811+-0.0000022}{\micro\second}$~\cite{PDG:muon} gegeben. Dies bedeutet eine Abweichung von $\qty{7.1}{\percent}$. 
Diese Abweichung lässt sich nicht mit der angegebenen -~rein statistischen~- Unsicherheit des Messwertes erklären. Eine potentielle Fehlerquelle
ist die Bildung myonischer Atome, die eine kürzere Lebensdauer als die kosmischen Myonen aufweisen, was die Abweichung zu kleineren Werten erklären könnte.
Ebenfalls werden statistische Unsicherheiten aus der Umrechnung der Kanalnummern in Messzeiten nicht inkludiert, 
da die verwendete Minimierung mittels \texttt{scipy} auf einer nicht-linearen Least-Squares Optimierung beruht, welche nur die Abweichungen in y-Richtung berücksichtigt.\\ 
Bei der Bestimmung der Kalibrierung der 
Koinzidenz wurde eine Halbwertsbreite von $\qty{14}{\nano\second}$ bestimmt. Allgemein lässt sich feststellen, dass die Messwerte zur Bestimmung der Halbwertsbreite 
starken Schwankungen unterliegen, deren Ursache nicht ermittelt werden konnte. Es lässt sich nur schwer ein Plateaubereich festlegen.
Eine genauere Messung mit längeren Messintervallen könnte aussagekräftigere Ergebnisse liefern.

\textcolor{red}{Es kann und soll aber mit eurer Erwartung zu der Breite in Verbindung gebracht werden.}
